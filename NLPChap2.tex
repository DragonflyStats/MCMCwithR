2.2 Introduction
People communicate in many different ways: through speaking and listening,
making gestures, using specialised hand signals (such as when driving or directing
traffic), using sign languages for the deaf, or through various forms of text.
By text we mean words that are written or printed on a flat surface (paper, card,
street signs and so on) or displayed on a screen or electronic device in order to be
read by their intended recipient (or by whoever happens to be passing by).

This course will focus only on the last of these: we will be concerned with various
ways in which computer systems can analyse and interpret texts, and we will assume
for convenience that these texts are presented in an electronic format. This is of
course quite a reasonable assumption, given the huge amount of text we can access
via the World Wide Web and the increasing availability of electronic versions of
newspapers, novels, textbooks and indeed subject guides. This chapter introduces
some essential concepts, techniques and terminology that will be applied in the rest
of the course. Some material in this chapter is a little technical but no programming
is involved at this stage.
%====================================================================================================%
2.3.2 Parts of speech
A further stage in analysing text is to associate every token with a grammatical
category or part of speech (POS). A number of different POS classifications have
been developed within computational linguistics and we will see some examples in
subsequent chapters. The following is a list of categories that are often encountered
in general linguistics: you will be familiar with many of them already from learning
the grammar of English or other languages, though some terms such as Determiner
or Conjunction may be new to you.

Noun fish, book, house, pen, procrastination, language
Proper noun John, France, Barack, Goldsmiths, Python
Verb loves, hates, studies, sleeps, thinks, is, has
Adjective grumpy, sleepy, happy, bashful
Adverb slowly, quickly, now, here, there
Pronoun I, you, he, she, we, us, it, they
Preposition in, on, at, by, around, with, without
Conjunction and, but, or, unless
Determiner the, a, an, some, many, few, 100
