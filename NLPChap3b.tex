3.5.2 British National Corpus
The BNC is created and managed by the BNC consortium, which includes Oxford
and Lancaster universities, dictionary publishers OUP, Longmans and Chambers, and
the British Library. It was developed between 1991 and 1994 and consists of 100
million words: 90 per cent written and 10 per cent transcriptions of speech. This
was one of the first corpora to include spontaneous spoken English. The corpus was
marked up using an automated part-of-speech tagger which resulted in a significant
saving of time and expense compared with manual annotation by competent
speakers of the language, but means that there is inevitably a degree of error – as
you may discover in the course of the exercise given later in this chapter.
You can access this corpus online and perform various kinds of analysis using the
Simple Query language. Registration is required via the following link but there is
currently no charge:
http://bncweb.lancs.ac.uk/bncwebSignup/user/login.php (last visited 27th
May 2013)
3.5.3 COBUILD Bank of English
The COBUILD project involved Collins Dictionaries and the University of
Birmingham. The Collins corpus is a 2.5-billion word analytical database of English.
It contains written material from websites, newspapers, magazines and books
published around the world, and spoken material from radio, TV and everyday
conversations. The Bank of EnglishTMforms part of the Collins Corpus and contains
650 million words. It was used as a basis for the Collins Advanced Learner’s
Dictionary, grammars and various tutorial materials for learners of English. It is not
included in the NLTK but there is limited online access via
http://www.collinslanguage.com/wordbanks.
34
Corpora
3.5.4 Penn Treebank
The Penn Treebank with its various offshoots is one of the widely used linguistic
resources among empirical researchers.
It includes a collection of texts in four different formats:
Raw text (original).
Tagged with POS using a tagset which was developed as part of the project.
‘Parsed’; that is, marked up with constituent structure.
Combined, including both POS tags and constituent structure.
The Penn Treebank project . . . has produced treebanks from the Brown, Switchboard, ATIS and Wall Street
Journal corpora of English, as well as treebanks in Arabic and Chinese.
Jurafsky and Martin (2009, p. 438)
The project began at the University of Pennsylvania in the 1990s and the results
have been used as a basis for further annotation efforts involving semantics and
rhetorical structure. The NLTK includes a selection from the Wall Street Journal
(WSJ) component of the Treebank, which can be accessed in each of the above
formats and additionally with a simplified POS tagset (Bird et al., 2009, Table 5-1,
p. 183). Here is an excerpt showing the four different formats:
Raw text
Pierre Vinken, 61 years old, will join the board as a
nonexecutive director Nov. 29.
Tagged
[ Pierre/NNP Vinken/NNP ]
,/,
[ 61/CD years/NNS ]
old/JJ ,/, will/MD join/VB
[ the/DT board/NN ]
as/IN
[ a/DT nonexecutive/JJ director/NN Nov./NNP 29/CD ]
./.
Parsed
( (S (NP-SBJ (NP Pierre Vinken)
,
(ADJP (NP 61 years) old)
35
CO3354 Introduction to natural language processing
,)
(VP will
(VP join
(NP the board)
(PP-CLR as
(NP a nonexecutive director))
(NP-TMP Nov. 29)))
.))
Combined
( (S
(NP-SBJ
(NP (NNP Pierre) (NNP Vinken) )
(, ,)
(ADJP
(NP (CD 61) (NNS years) )
(JJ old) )
(, ,) )
(VP (MD will)
(VP (VB join)
(NP (DT the) (NN board) )
(PP-CLR (IN as)
(NP (DT a) (JJ nonexecutive) (NN director) ))
(NP-TMP (NNP Nov.) (CD 29) )))
(. .) ))
3.5.5 Gutenberg archive
The NLTK includes a small selection of out-of-copyright literary texts from Project
Gutenberg, an archive of free electronic books hosted at http://www.gutenberg.org/
Some of the texts included are:
Jane Austen: Emma, Persuasion
GK Chesterton: Father Brown stories, The Man Who Was Thursday
William Blake: Poems
Milton: Paradise Lost
Shakespeare: Julius Caesar, Macbeth, Hamlet
3.5.6 Other corpora
Some further corpora included with the NLTK are:
The Reuters corpus distributed with the NLTK contains 10,788 news documents
totalling 1.3m words, partitioned into ‘training’ and ‘test’ sets. This split is for
training and testing machine learning algorithms: we return to this topic in
Chapter 5 of this subject guide.
36
Corpora
US Presidents’ inaugural and State of the Union addresses, organised as separate
files.
UN Declaration of Human Rights in 300+ languages. Here are a few excerpts:
 All human beings are born free and equal in dignity and rights.
 Abantu bonke bazalwa bekhululekile njalo belingana kumalungelo abo.
 Todos os seres humanos nascem livres e iguais em dignidade e em direitos.
Other corpora with online query interfaces include:
1. The Corpus of Contemporary American English, hosted at Brigham Young
University, is claimed to be ‘the only large and balanced corpus of American
English’. http://corpus.byu.edu/coca/ (last visited 27th May 2013)
2. The Intellitext project at the University of Leeds ‘aims to facilitate corpus use for
academics working in various areas of the humanities’ and currently provides
access to monolingual and parallel corpora in several European and Asian
languages. http://corpus.leeds.ac.uk/it/ (last visited 27th May 2013)
Learning activity
1. Pick two or three of the corpora mentioned above and research them online, focusing on questions
such as:
how large is the corpus?
what language(s) and genre(s) does it represent?
when was it constructed and what is its intended use?
what is the sampling frame?
what level of interannotator agreement was achieved, if reported?
2. Logon to the BNC Web (free registration needed) or another online corpus. Study the documentation
provided and search for data to answer the following questions:
What syntactic categories can the following words have? total, pancake, requisition, acquisition.
The guide to Simple Query Syntax provided with the BNC warns that ‘part-of-speech tags have
been assigned by an automatic software tool and are not always correct’. Have your answers to
the previous question shown up any examples of incorrect classification, in your view?
What prepositions can follow the verb talk? Give an example in each case.
3.5.7 WordNet
The NLTK also includes English WordNet, with 155,287 words and 117,659
synonym sets or synsets. A synset consists of a set of synonymous words paired with
a definition and linked to words with more general or more specific meanings. For
example, table has various meanings:
table.n.01 ['table', 'tabular\_array'], "a set of data arranged in
rows and columns"
table.n.02 ['table'], "a piece of furniture having a smooth flat top
37
CO3354 Introduction to natural language processing
that is usually supported by one or more vertical legs"
table.n.02 hyponyms: drop-leaf\_table, coffee\_table, pool\_table, altar,
table.n.02 hypernyms: furniture
3.6 Some basic corpus analysis
This chapter describes some relatively simple techniques, extracting various kinds of
data in suitable formats for human interpretation of the results. Chapters 4 and 5 of
the subject guide will look at ways the analysis and interpretation itself can be
automated to varying degrees.
Concordancing involves locating every instance of a word or phrase within a text or
corpus and presenting it in context, usually a fixed number of words before and
after each occurrence.
Collocations are pairs of sequences of words that occur together in a text more
frequently than would be expected by chance, and so provide a rough indication
of the content or style of a document.
Conditional frequency distributions support an elementary form of statistical
analysis. A frequency distribution counts observable events and a conditional
frequency distribution pairs each event with a condition. Some sample
applications are:
Comparing the use of particular words in different genres.
Comparing word lengths in different languages.
3.6.1 Frequency distributions
The following worked example displays some rudimentary stylistic analysis by
ranking the POS tags in a corpus according to frequency.
Calculating tag frequency
1. Import the Brown corpus.
2. List the different categories within the corpus.
3. Count the number of sentences in the science fiction category.
4. Extract all the word tokens from the science fiction category, paired with their
tags, and store them in the variable bsf. Note that the simplified tagset is
selected.
5. Calculate a frequency distribution of the tags: this gives an ordered list of the
tags paired with their frequency in the variable sf tag fd. (Only the 12 most
frequent are shown.)
>>> from nltk.corpus import brown
>>> brown.categories()
['adventure', 'belles_lettres', 'editorial', 'fiction',
'government', 'hobbies', 'humor', 'learned', 'lore',
38
Some basic corpus analysis
'mystery', 'news', 'religion', 'reviews', 'romance', 'science_fiction']
>>> sf_sents = brown.sents(categories = 'science_fiction')
>>> len(sf_sents)
948
>>> bsf = brown.tagged_words(categories = 'science_fiction',
simplify_tags=True)
>>> sf_tag_fd = nltk.FreqDist(tag for (word,tag) in bsf)
>>> sf_tag_fd.keys()
['N', 'V', 'DET', 'PRO', 'P', '.', 'ADJ', ',', 'CNJ', 'ADV', 'VD', 'NP', ]
>>> sf_tag_fd.tabulate()
N V DET PRO P . ADJ , CNJ ADV VD NP
2232 1473 1345 1268 1182 1078 793 791 685 644 531 467
Learning activity
1. Repeat the above process for other categories such as romance, news and religion. How do the
frequency distributions and sentence counts enable you to compare the literary styles of these
genres? Explain any assumptions you make.
2. Having read through Bird et al. (2009, section 2.1), with particular attention to Table 2-3 on page 50,
answer the following questions:
(a) Summarise the README file from the Reuters corpus.
(b) Create a variable soysents containing all sentences from reports concerning soy products.
(c) Display the first ten sentences in soysents.
(d) Create a variable metalwords containing all words from reports concerning metals.
(e) What are the most frequently mentioned metals in this collection? Caution: why might this result
be less than 100 per cent reliable?
3.6.2 DIY corpus: some worked examples
NLTK’s plain text corpus reader can be used to build a ‘corpus’ from a collection of
text files. The resulting corpus will be formatted for access as raw text, lists of words
or lists of sentences and can be re-formatted for other functions such as
concordancing and finding collocations.
The first example involves a one-text ‘corpus’ of the recent report from the UK
Equality and Human Rights Commission: How fair is Britain?
Step 1 Download the report as a PDF from http://www.equalityhumanrights.com
Step 2 Manually extract text using Adobe Acrobat or another PDF reader and save
as a .txt file
Step 3 Point the corpus reader to the directory where you have saved the text file.
>>> import nltk
>>> from nltk.corpus import PlaintextCorpusReader
>>> corpus_root = 'C:\NLP-stuff\Corpora'
>>> mycorpus = PlaintextCorpusReader(corpus_root,'.*')
39
CO3354 Introduction to natural language processing
We can now use the raw, words and sents methods to display the content in
different formats:
>>> mycorpus.fileids()
['howfair.txt']
>>> mycorpus.words('howfair.txt')
['Equality', 'and', 'Human', 'Rights', 'Commission', ]
>>> hf_raw = mycorpus.raw('howfair.txt')
>>> hf_raw[:100]
'Equality and Human Rights Commission\r\nTriennial
Review 2010\r\nExecutive Summary\r\nHow fair\r\nis Britain'
>>> mycorpus.sents('howfair.txt')
[['Equality', 'and', 'Human', 'Rights', 'Commission', 'Triennial',
'Review', '2010', 'Executive', 'Summary', 'How', 'fair', 'is',
'Britain', '?'], ...
Concordancing and collocations
The Text method formats the content in a way which can be accessed by the
concordance and collocation methods. Note that concordancing will always
return fixed-length strings which include your target text as a substring, and so may
be cut off in the middle of a word.
>>> fair=nltk.Text(mycorpus.words('howfair.txt'))
>>> fair.concordance('equal')
Building index...
Displaying 3 of 3 matches:
has narrowed considerably since the equal Pay Act 1970 came into force in 1975
sonal circumstances , should have an equal opportunity to have a say in decisio
that every individual should have an equal chance to make the most of their tal
>>> fair.collocations()
Building collocations list
Human Rights; Rights Commission; Significant findings; Headline data;
Executive Summary; less likely; ethnic minority; life expectancy;
0845 604; domestic violence; hate crime; labour market; disabled people;
mental health; Black Caribbean; different groups; religiously motivated;
sexual assault; minority groups; formal childcare
Conditional frequency distribution
Recall that a frequency distribution is a set of ordered pairs < event; count >
where count is the number of times event occurs. In our context event is a word-type
and count is the number of tokens of that type in a text. A conditional frequency
distribution is a collection of frequency distributions, each one for a different
condition.
For this example we add a second document to the corpus, extracted from a PDF
‘Guide to data protection’.
Step 1 Create a single variable text word consisting of pairs of each word-token
with the fileid of the document it occurs in.
40
Summary
Step 2 Create a conditional frequency distribution, which will tell you the frequency
of each word in both texts.
Step 3 Pick a sample of words which are likely to occur in both documents, and
tabulate their comparative frequency.
>>> text_word = [(text,word) for text in ['howfair.txt','guide.txt']
for word in mycorpus.words(text)]
>>> text_word[:10]
[('howfair.txt', 'Equality'), ('howfair.txt', 'and'),
('howfair.txt', 'Human'),('howfair.txt', 'Rights'),
('howfair.txt', 'Commission'), ('howfair.txt', 'Triennial'),
('howfair.txt', 'Review'), ('howfair.txt', '2010'),
('howfair.txt', 'Executive'),
('howfair.txt', 'Summary')]
>>> cfd = nltk.ConditionalFreqDist(text_word)
>>> cfd
<ConditionalFreqDist with 2 conditions>
>>> cfd.conditions()
['guide.txt', 'howfair.txt']
>>> cfd['howfair.txt']
<FreqDist with 16391 outcomes>
>>> cfd['guide.txt']
<FreqDist with 47064 outcomes>
Testing the CFD
>>> cfd['guide.txt']['fair']
31
>>> cfd['howfair.txt']['fair']
30
>>> keywords = ['fair','police','crime','office','equal','privacy']
>>>>cfd.tabulate(conditions=['howfair.txt','guide.txt'],
samples=keywords)
fair police crime office equal privacy
howfair.txt 30 15 29 4 2 0
guide.txt 31 16 17 2 0 26
Learning activity
Find some suitable electronic documents and follow the above techniques to construct a ‘mini-corpus’. The
documents in these examples were sourced from UK government websites: you may find similar
documents on the website of your own country’s government, or of transnational organisations like the
European Union or the United Nations. Think of some terms which are likely to occur in several of these
documents and compare them using a conditional frequency distribution. If you can find a lengthy report
which is issued along with a shorter summary, it is an interesting exercise to pick some key terms and see if
their comparative frequency is the same or similar in the original document and the summary.
3.7 Summary
1. A corpus is a collection of linguistic data which was not originally produced for
the purposes of linguistic analysis. Properly speaking it should be constructed so
as to be balanced and representative. If a corpus includes any kind of
41
CO3354 Introduction to natural language processing
annotation, it is good practice to use multiple annotators for at least a sample of
the corpus and report the level of inter-annotator agreement that was
achieved.
2. Some uses of corpora include:
Lexicography (compiling dictionaries).
Compiling grammars for education and reference purposes.
Stylistics: developing techniques to identify the author or genre of a
document; investigating the effect on language use of different channels
such as email, chat, face-to-face conversation, telephone calls and
hand-written letters.
Training and evaluation in linguistic research, using machine learning
techniques.
3. This chapter includes brief descriptions of several well-known and widely-used
corpora such as the Brown corpus, the BNC and the Penn Treebank.
4. Students on this course can access a variety of corpora through online interfaces
or by using corpus tools provided with the NLTK.
5. Some simple techniques for analysing corpora include concordancing,
collocations and (conditional) frequency distributions. None of these techniques
involve automated linguistic analysis: the interpretation of the outputs is down
to the analyst.
3.8 Sample examination question
a) Explain what is meant by the following types of corpus, and describe an example
in each category that you have encountered during this course:
isolated
categorised
overlapping
temporal.
b) What applications would a tagged and parsed corpus be suitable for? What are
some advantages and disadvantages of using an automated tagger to build such a
corpus?
c) Suppose the following lists show the number of sentences and the most commonly
occurring part-of-speech tags in three different categories of text in a corpus, with
their frequency of occurrence in brackets. What can you say about the styles of these
documents from studying these results? Discuss any assumptions you make.
Category A (4623 sentences) N (22236) P (10845) DET (10648) NP (8336) V
(7346) ADJ (5435) ‘,’ (5188) ‘.’ (4472) CNJ (4227) PRO (3408) ADV (2770) VD
(2531) . . .
Category B (948 sentences) N (2232) V (1473) DET (1345) PRO (1268) P (1182)
‘.’ (1078) ADJ (793) ‘,’ (791) CNJ (685) ADV (644) VD (531) NP (467) . . .
Category C (1716 sentences) N (7304) P (4364) DET (4239) V (3863) ADJ (2620)
CNJ (2324) PRO (2243) ‘,’ (1913) ‘.’ (1892) ADV (1485) NP (1224) VN (952)
. . .
42
Appendix C
Answers to selected activities
This section includes model solutions to some of the exercises and activities where
appropriate. In other cases there is no ‘correct’ answer and the point of the activity is
to stimulate you to engage in independent self-directed learning.
Chapter 2: Introducing NLP: patterns and structure in natural
language
Identify parts of speech, page 14
Jack (Proper Noun) and (conjunction) Jill (Proper Noun) went (verb) up
(preposition) the (determiner) hill (noun).
The (determiner) owl (noun) and (conjunction) the (determiner) pussycat (noun)
went (verb) to (preposition) sea (noun).
Operation of a finite-state machine, page 17
1. John swam.
2. (a) John and Mary walked on the hill.
(b) The cat sat on the mat and slept.
(c) John or a fish walked on a hill and barked.
(d) . . .
Coding regular expressions, page 19
1. a(aa)*(bb)*
2. (aaa)j(aab)j(abb)j(bbb)j(bba)j(baa)j(aba)j(bab)
97
CO3354 Introduction to natural language processing
Regular grammars, page 21
S!either j if S1
S!the j a j one S2
S2!happy S2
S2!(boy j girl j dog) eats (‘ice cream’ j ‘hot dogs’ j candy) S3
S3!or j then S2
S3!
This is a slightly idealised rendering of Pinker’s state diagram which appears to have
no halting state.
The problem with this grammar can be clearly seen: the rule S3 ! or j then S2 has
no connection with the rule that introduces either or if and so there is no way to
ensure that the appropriate conjunction is used.
Past tense forms, page 25
The general idea is that rules need to be conditional in order to handle
non-standard cases before applying general regularities: so a reasonable rule based
on this data set might be:
welcome ! welcomed else
-come ! -came
Chapter 3: Getting to grips with natural language data
Online corpus queries, page 37
Examples of incorrect tagging: search on to total/V gives examples like:
. . . a ticket to total oblivion.
. . . to describe it to total strangers.
as well as ‘correct’ examples like
. . . thought to total about 1,500 families . . .
. . . a great opportunity to total and celebrate all the small wins made over the
year
98
APPENDIX C. ANSWERS TO SELECTED ACTIVITIES
Using NLTK tools, page 39
1. ‘Stylistic’ analysis
Science Fiction (948 sentences) N (2232) V (1473) DET (1345) PRO (1268) P
(1182) ‘.’ (1078) ADJ (793) ‘,’ (791) CNJ (685) ADV (644) VD (531) NP (467)
. . .
News (4623 sentences) N (22236) P (10845) DET (10648) NP (8336) V (7346)
ADJ (5435) ‘,’ (5188) ‘.’ (4472) CNJ (4227) PRO (3408) ADV (2770) VD (2531)
. . .
Religion (1716 sentences) N (7304) P (4364) DET (4239) V (3863) ADJ (2620)
CNJ (2324) PRO (2243) ‘,’ (1913) ‘.’ (1892) ADV (1485) NP (1224) VN (952)
. . .
Some counts:
SF: N 2232, PRO 1268, NP 467 ADJ 793
NE: N 22236, NP 8336, PRO 3408 ADJ 5435
RE: N 7304, PRO 2243 NP 1224 ADJ 2620
SF: S 948, COMMA 791, CNJ 685
NE: COM 5188 S 4623 CNJ 4227
RE: CNJ 2324 COM 1913 S 1716
Some tentative conclusions:
1. Reference and description: both SF and RE use pronouns more than proper
names; News has more proper names. Hard to interpret without further
analysis: if an SF work includes a lot of dialogue for example, it might be more
natural for characters to refer to each other as I, we, you and so on rather than
by name. And news stories tend to be about named individuals.
2. Syntactic complexity: we cannot be very precise with this data but it looks as if
the SF genre has the least syntactic complexity and the RE genre the highest,
judging from the numbers of commas and conjunctions per sentence. Of course
we cannot tell whether these tokens are connecting clauses or other types of
phrases.
In an examination or coursework question, you would get credit for discussing these
and other characteristics in the light of your impressionistic understanding of the
typical styles for these genres.
2. Reuters
Display the README file from the Reuters corpus.
from nltk.corpus import reuters
desc = reuters.readme()
print desc
99
CO3354 Introduction to natural language processing
Create a variable soysents containing all sentences from reports concerning soy
products.
reuters.categories()
Pick out categories relating to soy:
soysents = reuters.sents(categories=['soy-meal', ...])
Display the first ten sentences in soysents.
print soysents[:10]
Create a variable metalwords containing all words from reports concerning
metals.
metalwords = reuters.words(categories = ['alum','copper','gold', ...])
(Note that inspection of texts in the alum category confirms that they are about
aluminium.)
What are the most frequently mentioned metals in this collection? Caution: why
might this result be less than 100 per cent reliable?
from nltk.book import *
freqmetal = FreqDist(metalwords)
freqmetalkeys = freqmetal.keys()
freqmetal[:100]
Remember that the contents of a frequency distribution are listed in the order of
their frequency of occurrence. By scanning the output you should see that the
first three metals listed are gold, copper and steel. However caution is in order
as Reuters is an overlapping corpus, so we may be double-counting some
occurrences. These metals may also be mentioned under the category
strategic-metal, or some reports may mention more than one kind of metal
and so come under multiple categories.
