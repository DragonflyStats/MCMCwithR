Introduction to natural
language processing
R. Kibble
CO3354
2013
Undergraduate study in
Computing and related programmes
This is an extract from a subject guide for an undergraduate course offered as part of the
University of London International Programmes in Computing. Materials for these programmes
are developed by academics at Goldsmiths.
For more information, see: www.londoninternational.ac.uk
This guide was prepared for the University of London International Programmes by:
R. Kibble
This is one of a series of subject guides published by the University. We regret that due to pressure of work the author is
unable to enter into any correspondence relating to, or arising from, the guide. If you have any comments on this subject
guide, favourable or unfavourable, please use the form at the back of this guide.
University of London International Programmes
Publications Office
32 Russell Square
London WC1B 5DN
United Kingdom
www.londoninternational.ac.uk
Published by: University of London
Copyright © Department of Computing, Goldsmiths 2013
The University of London and Goldsmiths assert copyright over all material in this subject guide except where otherwise
indicated. All rights reserved. No part of this work may be reproduced in any form, or by any means, without permission in
writing from the publisher. We make every effort to respect copyright. If you think we have inadvertently used your copyright
material, please let us know.
Contents
Preface 1
About this half unit . . . . . . . . . . . . . . . . . . . . . . . . . . . . . . . . 1
Assessment . . . . . . . . . . . . . . . . . . . . . . . . . . . . . . . . . . . . 1
The subject guide and other learning resources . . . . . . . . . . . . . . . . 2
Suggested study time . . . . . . . . . . . . . . . . . . . . . . . . . . . . . . . 2
Acknowledgement . . . . . . . . . . . . . . . . . . . . . . . . . . . . . . . . 3
1 Introduction: how to use this subject guide 5
1.1 Introduction . . . . . . . . . . . . . . . . . . . . . . . . . . . . . . . . . 5
1.2 Aims of the course . . . . . . . . . . . . . . . . . . . . . . . . . . . . . 5
1.3 Learning outcomes . . . . . . . . . . . . . . . . . . . . . . . . . . . . . 6
1.4 Reading list and other learning resources . . . . . . . . . . . . . . . . . 6
1.5 Software requirements . . . . . . . . . . . . . . . . . . . . . . . . . . . 8
1.6 How to use the guide/structure of the course . . . . . . . . . . . . . . 8
1.6.1 Chapter 2: Introducing NLP: patterns and structures in language 8
1.6.2 Chapter 3: Getting to grips with natural language data . . . . . 8
1.6.3 Chapter 4: Computational tools for text analysis . . . . . . . . . 9
1.6.4 Chapter 5: Statistically-based techniques for text analysis . . . 9
1.6.5 Chapter 6: Analysing sentences: syntax and parsing . . . . . . 9
1.6.6 Appendices . . . . . . . . . . . . . . . . . . . . . . . . . . . . . 9
1.7 What the course does not cover . . . . . . . . . . . . . . . . . . . . . . 9
2 Introducing NLP: patterns and structure in language 11
Essential reading . . . . . . . . . . . . . . . . . . . . . . . . . . . . . . 11
Recommended reading . . . . . . . . . . . . . . . . . . . . . . . . . . . 11
Additional reading . . . . . . . . . . . . . . . . . . . . . . . . . . . . . 11
2.1 Learning outcomes . . . . . . . . . . . . . . . . . . . . . . . . . . . . . 11
2.2 Introduction . . . . . . . . . . . . . . . . . . . . . . . . . . . . . . . . . 12
2.3 Basic concepts . . . . . . . . . . . . . . . . . . . . . . . . . . . . . . . . 12
2.3.1 Tokenised text and pattern matching . . . . . . . . . . . . . . . 12
Activity: Recognising names . . . . . . . . . . . . . . . . . . . . . . . . 13
2.3.2 Parts of speech . . . . . . . . . . . . . . . . . . . . . . . . . . . 13
Activity: identify parts of speech . . . . . . . . . . . . . . . . . . . . . 14
2.3.3 Constituent structure . . . . . . . . . . . . . . . . . . . . . . . . 14
Activity: Writing production rules . . . . . . . . . . . . . . . . . . . . . 15
2.4 A closer look at syntax . . . . . . . . . . . . . . . . . . . . . . . . . . . 15
2.4.1 Operation of a finite-state machine . . . . . . . . . . . . . . . . 16
Activity: Finite-state machines . . . . . . . . . . . . . . . . . . . . . . . 17
2.4.2 Representing finite-state machines . . . . . . . . . . . . . . . . 17
2.4.3 Declarative alternatives to finite-state machines . . . . . . . . . 18
Activity: Coding regular expressions . . . . . . . . . . . . . . . . . . . 19
Activity: tree diagrams for a regular language . . . . . . . . . . . . . . 21
2.4.4 Limitations of finite-state methods – introducing context-free
grammars . . . . . . . . . . . . . . . . . . . . . . . . . . . . . . 21
Activity: Regular grammars . . . . . . . . . . . . . . . . . . . . . . . . 21
Activity: Context-free grammar . . . . . . . . . . . . . . . . . . . . . . 23
2.4.5 Looking ahead: some further uses of regular expressions . . . . 23
i
CO3354 Introduction to natural language processing
2.4.6 Looking ahead: grammars and parsing . . . . . . . . . . . . . . 24
2.5 Word structure . . . . . . . . . . . . . . . . . . . . . . . . . . . . . . . 24
Activity: Past tense formation . . . . . . . . . . . . . . . . . . . . . . . 25
2.6 A brief history of natural language processing . . . . . . . . . . . . . . 25
2.7 Summary . . . . . . . . . . . . . . . . . . . . . . . . . . . . . . . . . . 27
2.8 Sample examination questions . . . . . . . . . . . . . . . . . . . . . . 27
3 Getting to grips with natural language data 29
Essential reading . . . . . . . . . . . . . . . . . . . . . . . . . . . . . . 29
Recommended reading . . . . . . . . . . . . . . . . . . . . . . . . . . . 29
Additional reading . . . . . . . . . . . . . . . . . . . . . . . . . . . . . 29
3.1 Learning outcomes . . . . . . . . . . . . . . . . . . . . . . . . . . . . . 29
3.2 Using the Natural Language Toolkit . . . . . . . . . . . . . . . . . . . . 29
3.3 Corpora and other data resources . . . . . . . . . . . . . . . . . . . . . 30
3.4 Some uses of corpora . . . . . . . . . . . . . . . . . . . . . . . . . . . . 31
3.4.1 Lexicography . . . . . . . . . . . . . . . . . . . . . . . . . . . . 32
3.4.2 Grammar and syntax . . . . . . . . . . . . . . . . . . . . . . . . 32
3.4.3 Stylistics: variation across authors, periods, genres and channels
of communication . . . . . . . . . . . . . . . . . . . . . . . 32
3.4.4 Training and evaluation . . . . . . . . . . . . . . . . . . . . . . 33
3.5 Corpora . . . . . . . . . . . . . . . . . . . . . . . . . . . . . . . . . . . 33
3.5.1 Brown corpus . . . . . . . . . . . . . . . . . . . . . . . . . . . . 34
3.5.2 British National Corpus . . . . . . . . . . . . . . . . . . . . . . 34
3.5.3 COBUILD Bank of English . . . . . . . . . . . . . . . . . . . . . 34
3.5.4 Penn Treebank . . . . . . . . . . . . . . . . . . . . . . . . . . . 35
3.5.5 Gutenberg archive . . . . . . . . . . . . . . . . . . . . . . . . . 36
3.5.6 Other corpora . . . . . . . . . . . . . . . . . . . . . . . . . . . . 36
Activity: Online corpus queries . . . . . . . . . . . . . . . . . . . . . . 37
3.5.7 WordNet . . . . . . . . . . . . . . . . . . . . . . . . . . . . . . . 37
3.6 Some basic corpus analysis . . . . . . . . . . . . . . . . . . . . . . . . 38
3.6.1 Frequency distributions . . . . . . . . . . . . . . . . . . . . . . 38
Activity: Using NLTK tools . . . . . . . . . . . . . . . . . . . . . . . . . 39
3.6.2 DIY corpus: some worked examples . . . . . . . . . . . . . . . 39
Activity: building and analysing a DIY corpus . . . . . . . . . . . . . . 41
3.7 Summary . . . . . . . . . . . . . . . . . . . . . . . . . . . . . . . . . . 41
3.8 Sample examination question . . . . . . . . . . . . . . . . . . . . . . . 42
4 Computational tools for text analysis 43
Essential reading . . . . . . . . . . . . . . . . . . . . . . . . . . . . . . 43
Recommended reading . . . . . . . . . . . . . . . . . . . . . . . . . . . 43
Additional reading . . . . . . . . . . . . . . . . . . . . . . . . . . . . . 43
4.1 Introduction and learning outcomes . . . . . . . . . . . . . . . . . . . 43
4.1.1 Learning outcomes . . . . . . . . . . . . . . . . . . . . . . . . . 43
4.2 Data structures . . . . . . . . . . . . . . . . . . . . . . . . . . . . . . . 44
Activity: strings and sequences . . . . . . . . . . . . . . . . . . . . . . 44
4.3 Tokenisation . . . . . . . . . . . . . . . . . . . . . . . . . . . . . . . . . 44
4.3.1 Some issues with tokenisation . . . . . . . . . . . . . . . . . . . 45
4.3.2 Tokenisation in the NLTK . . . . . . . . . . . . . . . . . . . . . 46
Activity: Tokenising text . . . . . . . . . . . . . . . . . . . . . . . . . . 46
4.4 Stemming . . . . . . . . . . . . . . . . . . . . . . . . . . . . . . . . . . 46
Activity: Comparing stemmers . . . . . . . . . . . . . . . . . . . . . . . 48
4.5 Tagging . . . . . . . . . . . . . . . . . . . . . . . . . . . . . . . . . . . 48
4.5.1 RE tagging . . . . . . . . . . . . . . . . . . . . . . . . . . . . . 49
Activity: Tagging with REs . . . . . . . . . . . . . . . . . . . . . . . . . 51
4.5.2 Trained taggers and backoff . . . . . . . . . . . . . . . . . . . . 51
ii
4.5.3 Transformation-based tagging . . . . . . . . . . . . . . . . . . . 53
4.5.4 Evaluation and performance . . . . . . . . . . . . . . . . . . . . 53
Activity: Trained taggers . . . . . . . . . . . . . . . . . . . . . . . . . . 53
4.6 Summary . . . . . . . . . . . . . . . . . . . . . . . . . . . . . . . . . . 53
4.7 Sample examination question . . . . . . . . . . . . . . . . . . . . . . . 54
5 Statistically-based techniques for text analysis 57
Essential reading . . . . . . . . . . . . . . . . . . . . . . . . . . . . . . 57
Recommended reading . . . . . . . . . . . . . . . . . . . . . . . . . . . 57
Additional reading . . . . . . . . . . . . . . . . . . . . . . . . . . . . . 57
5.1 Learning outcomes . . . . . . . . . . . . . . . . . . . . . . . . . . . . . 57
5.2 Introduction . . . . . . . . . . . . . . . . . . . . . . . . . . . . . . . . . 58
5.3 Some fundamentals of machine learning . . . . . . . . . . . . . . . . . 58
5.3.1 Naive Bayes classifiers . . . . . . . . . . . . . . . . . . . . . . . 58
Activity: Bayes’ rule . . . . . . . . . . . . . . . . . . . . . . . . . . . . 59
5.3.2 Hidden Markov models . . . . . . . . . . . . . . . . . . . . . . 60
5.3.3 Information and entropy . . . . . . . . . . . . . . . . . . . . . . 61
5.3.4 Decision trees and maximum entropy classifiers . . . . . . . . . 62
Activity: further reading . . . . . . . . . . . . . . . . . . . . . . . . . . 63
5.3.5 Evaluation . . . . . . . . . . . . . . . . . . . . . . . . . . . . . . 63
5.4 Machine learning in action: document classification . . . . . . . . . . . 64
5.4.1 Summary: document classification . . . . . . . . . . . . . . . . 65
Activity: document classification . . . . . . . . . . . . . . . . . . . . . 66
5.5 Machine learning in action: information extraction . . . . . . . . . . . 66
5.5.1 Types of information extraction . . . . . . . . . . . . . . . . . . 67
5.5.2 Regular expressions for personal names . . . . . . . . . . . . . 67
Activity: coding regular expressions for proper names . . . . . . . . . . 69
5.5.3 Information extraction as sequential classification: chunking
and NE recognition . . . . . . . . . . . . . . . . . . . . . . . . . 69
Activity: chunking and NE recognition . . . . . . . . . . . . . . . . . . 71
5.6 Limitations of statistical methods . . . . . . . . . . . . . . . . . . . . . 71
5.7 Summary . . . . . . . . . . . . . . . . . . . . . . . . . . . . . . . . . . 72
5.8 Sample examination question . . . . . . . . . . . . . . . . . . . . . . . 72
6 Analysing sentences: syntax and parsing 75
Essential reading . . . . . . . . . . . . . . . . . . . . . . . . . . . . . . 75
Recommended reading . . . . . . . . . . . . . . . . . . . . . . . . . . . 75
Additional reading . . . . . . . . . . . . . . . . . . . . . . . . . . . . . 75
6.1 Learning outcomes . . . . . . . . . . . . . . . . . . . . . . . . . . . . . 75
6.2 Grammars and parsing . . . . . . . . . . . . . . . . . . . . . . . . . . . 75
6.3 Complicating CFGs . . . . . . . . . . . . . . . . . . . . . . . . . . . . . 76
6.3.1 Verb categories . . . . . . . . . . . . . . . . . . . . . . . . . . . 76
Activity: Verb categories . . . . . . . . . . . . . . . . . . . . . . . . . . 78
6.3.2 Agreement . . . . . . . . . . . . . . . . . . . . . . . . . . . . . 78
Activity: feature-based grammar . . . . . . . . . . . . . . . . . . . . . 80
6.3.3 Unbounded dependencies . . . . . . . . . . . . . . . . . . . . . 80
6.3.4 Ambiguity and probabilistic grammars . . . . . . . . . . . . . . 82
Activity: probabilistic grammar . . . . . . . . . . . . . . . . . . . . . . 85
6.4 Parsing . . . . . . . . . . . . . . . . . . . . . . . . . . . . . . . . . . . . 85
6.4.1 Recursive descent parsing . . . . . . . . . . . . . . . . . . . . . 86
6.4.2 Shift-reduce parsing . . . . . . . . . . . . . . . . . . . . . . . . 87
6.4.3 Parsing with a well-formed substring table . . . . . . . . . . . . 87
6.4.4 Finite-state machines and context-free parsing . . . . . . . . . . 89
Activity: Parsing . . . . . . . . . . . . . . . . . . . . . . . . . . . . . . 90
6.5 Summary . . . . . . . . . . . . . . . . . . . . . . . . . . . . . . . . . . 90
iii
CO3354 Introduction to natural language processing
6.6 Sample examination question . . . . . . . . . . . . . . . . . . . . . . . 91
A Bibliography 93
B Glossary 95
C Answers to selected activities 97
Chapter 2: Introducing NLP: patterns and structure in natural language . . . 97
Identify parts of speech, page 14 . . . . . . . . . . . . . . . . . . . . . 97
Operation of a finite-state machine, page 17 . . . . . . . . . . . . . . . 97
Coding regular expressions, page 19 . . . . . . . . . . . . . . . . . . . 97
Regular grammars, page 21 . . . . . . . . . . . . . . . . . . . . . . . . 98
Past tense forms, page 25 . . . . . . . . . . . . . . . . . . . . . . . . . 98
Chapter 3: Getting to grips with natural language data . . . . . . . . . . . . 98
Online corpus queries, page 37 . . . . . . . . . . . . . . . . . . . . . . 98
Using NLTK tools, page 39 . . . . . . . . . . . . . . . . . . . . . . . . . 99
Chapter 4: Computational tools for text analysis . . . . . . . . . . . . . . . . 100
Comparing stemmers, page 48 . . . . . . . . . . . . . . . . . . . . . . . 100
Tagging with REs, page 51 . . . . . . . . . . . . . . . . . . . . . . . . . 101
Chapter 5: Statistically-based techniques for text analysis . . . . . . . . . . . 101
Activity: Bayes’ Rule, page 59 . . . . . . . . . . . . . . . . . . . . . . . 101
Chapter 6: Analysing sentences: syntax and parsing . . . . . . . . . . . . . . 102
Activity: Verb categories, page 78 . . . . . . . . . . . . . . . . . . . . . 102
Activity: Feature-based grammar, page 80 . . . . . . . . . . . . . . . . 102
D Trace of recursive descent parse 105
E Sample examination paper with answering guidelines 107
E.1 Sample examination questions . . . . . . . . . . . . . . . . . . . . . . 108
E.2 Answering guidelines for sample examination questions . . . . . . . . 113
iv
Preface
About this half unit
This half unit course combines a critical introduction to key topics in theoretical and
computational linguistics with hands-on practical experience of using existing
software tools and developing applications to process texts and access linguistic
resources. The aims of the course and learning outcomes are listed in Chapter 1.
This course has no specific prerequisites. There will be some programming involved
and you will need to acquire some familiarity with the Python language, but you will
not be expected to develop substantial original code or to encode specialised
algorithms. The course involves some statistical techniques, but the only
mathematical knowledge assumed is an understanding of elementary probability
and familiarity with the concept of logarithms.
Before the advent of the world wide web, most machine-readable information was
stored in structured databases and accessed via specialised query languages such as
Structured Query Language (SQL). Nowadays the situation is reversed: most
information is found in unstructured or semi-structured natural language documents
and there is increasing demand for techniques to ‘unlock’ this data. Computing
graduates with knowledge of natural language processing techniques are finding
employment in areas such as text analytics, sentiment analysis, topic detection and
information extraction.
Assessment
The course is assessed via an unseen written examination. A sample examination
paper is provided in the Appendix at the end of this subject guide, with some
guidelines on how to answer the questions. You will be required to attempt three
questions out of a choice of five. The questions will cover ‘book knowledge’, problem
solving and short essays on more theoretical topics. The examination is not a
memory test but will be designed to assess your understanding of the course
content. There will also be coursework which will include a similar mix of questions,
but with a stronger focus on practical problem-solving.
You will be expected to provide electronic copies of your coursework for plagiarism
checking purposes. It is very important that any material that is not original to you
should be properly attributed and placed in quotation marks, with a full list of
references at the end of your submission. You should follow the style used in this
subject guide for citing references, for example:
Segaran (2007, pp.117–118) discusses some problems with rule-based spam filters.
Answers which consist entirely or mostly of quoted material are unlikely to get many
marks even if properly attributed, as simply reproducing an answer in someone
else’s words does not demonstrate that you have fully understood the material.
In order to give you some practice in problem-solving and writing short essays, there
1
CO3354 Introduction to natural language processing
are a number of Activities throughout this subject guide. The Appendix includes a
section ‘Answers to selected activities’, although these will not always provide
complete answers to the questions but are intended to indicate how particular types
of questions should be approached. Sample examination questions are provided at
the end of each chapter. Some, but not all, of these are included in the sample
examination paper with suggested answers at the end of the guide.
The subject guide and other learning resources
This subject guide is not intended as a self-contained textbook but sets out specific
topics for study in the CO3354 half unit. There is a recommended textbook and a
number of other readings are listed at appropriate places. There are also links to
websites providing useful resources such as software tools and access to online
linguistic data. The learning outcomes listed in the next chapter assume that you are
working through the recommended readings, activities and sample examination
questions. It will not be possible to pass this half unit by reading only the subject
guide. Please refer to the Computing VLE for other resources, which should be used
as an aid to your learning.
Suggested study time
The Student Handbook states that ‘To be able to gain the most benefit from the
programme, it is likely that you will have to spend at least 300 hours studying for
each full unit, though you are likely to benefit from spending up to twice this time’.
Note that this subject is a half unit.
The course is designed to be delivered over a ten-week term as one of four
concurrent modules, and this guide has six chapters. Chapter 1 goes into more detail
about the structure of the guide and the course, while Chapters 2 to 6 are each
dedicated to a particular topic. It is suggested that you spend about two weeks on
Chapters 1 and 2 together and each of Chapters 3 to 6, including the associated
reading and web-based material, and work through the activities and sample
examination questions during this time.
2
Contents
Acknowledgement
This subject guide draws closely on:
Bird, S., E. Klein and E. Loper, Natural Language Processing with Python. (O’Reilly
Media 2009) [ISBN 9780596516499; http://nltk.org/book].
You will be expected to draw on it in your studies and to use the accompanying
software package, the Natural Language Toolkit, which requires the Python
language. Natural language processing with Python has been made available under
the terms of the Creative Commons Attribution Noncommercial No-Derivative-Works
3.0 US License: http://creativecommons.org/licenses/by-nc-nd.3.0/us/legalcode (last
visited 13th April 2013).
3
CO3354 Introduction to natural language processing
4
Chapter 1
Introduction: how to use this subject guide
1.1 Introduction
The idea of computers being able to understand ordinary languages and hold
conversations with human beings has been a staple of science fiction since the first
half of the twentieth century and was envisaged in a classic paper by Alan Turing
(1950) as a hallmark of computational intelligence. Since the start of the
twenty-first century this vision has been starting to look more plausible: artificial
intelligence techniques allied with the scientific study of language have emerged
from universities and research laboratories to inform a variety of industrial and
commercial applications. Many websites now offer automatic translation; mobile
phones can appear to understand spoken questions and commands; search engines
like Google use basic linguistic techniques for automatically completing or
‘correcting’ your queries and for finding relevant results that are closely matched to
your search terms. We are still some way from full machine understanding of natural
language, however. Automated translations still need to be reviewed and edited by
skilled human translators while no computer system has yet come close to passing
the ‘Turing Test’ of convincingly simulating human conversation. Indeed it has been
argued that the Turing Test is a blind alley and that research should focus on
producing effective applications for specific requirements without seeking to
generate an illusion that users are interacting with a human rather than a machine
(Hayes and Ford, 1995). Hopefully, by the time you finish this course you will have
come to appreciate some of the challenges posed by full understanding of natural
language as well as the very real achievements that have resulted from focusing on a
range of specific, well-defined tasks.
1.2 Aims of the course
This course combines a critical introduction to key topics in theoretical linguistics
with hands-on practical experience of developing applications to process texts and
access linguistic resources. The main topics covered are:
accessing text corpora and lexical resources
processing raw text
categorising and tagging
extracting information from text
analysing sentence structure.
5
CO3354 Introduction to natural language processing
1.3 Learning outcomes
On successful completion of this course, including recommended readings, exercises
and activities, you should be able to:
1. utilise and explain the function of software tools such as corpus readers,
stemmers, taggers and parsers
2. explain the difference between regular and context-free grammars and define
formal grammars for fragments of a natural language
3. critically appraise existing Natural Language Processing (NLP) applications such
as chatbots and translation systems
4. describe some applications of statistical techniques to natural language analysis,
such as classification and probabilistic parsing.
Each main chapter contains a list of learning outcomes specific to that chapter at the
beginning, as well as a summary at the end of the chapter.
1.4 Reading list and other learning resources
This is a list of textbooks and other resources which will be useful for all or most
parts of the course. Additional readings will be given at the start of each chapter. See
the bibliography for a full list of books and articles referred to, including all ISBNs.
In some cases several different books will be listed: you are not expected to read all
of them, rather the intention is to give you some alternatives in case particular texts
are hard to obtain.
Essential reading
Bird, Klein, and Loper (2009): Natural Language Processing with Python. The full
text including diagrams is freely available online at http://nltk.org/book (last
visited 13th April 2013). The main textbook for this course, Natural Language
Processing with Python is the outcome of a project extending over several years
to develop the Natural Language Toolkit (NLTK), which is a set of tools and
resources for teaching computational linguistics. The NLTK comprises a suite of
software modules written in Python and a collection of corpora and other
resources. See section 1.5 below for advice on installing the NLTK and other
software packages.
In the course of working through this text you will gain some experience and
familiarity with the Python language, though you will not be expected to
produce substantial original code as part of the learning outcomes of the course.
Recommended reading
Pinker (2007). The Language Instinct. This book is aimed at non-specialists and
deals with many psychological and cultural aspects of language. Chapter 4 is
particularly relevant to this course as it provides a clear and accessible
presentation of two standard techniques for modelling linguistic structure:
finite-state machines and context-free grammars (though Pinker does not in fact
use these terms, as we will see in Chapter 2 of the subject guide).
6
Reading list and other learning resources
Jurafsky and Martin (2009): Speech and Language Processing, second edition.
Currently the definitive introductory textbook in this field, covering the major
topics in a way which combines theoretical issues with presentations of key
technologies, formalisms and mathematical techniques. Much of this book goes
beyond what you will need to pass this course, but it is always worth turning to
if you’re looking for a more in-depth discussion of any particular topics.
Perkins (2010): Python Text Processing with NLTK 2.0 Cookbook. This book will be
suitable for students who want to get more practice in applying Python
programming to natural language processing. Perkins explains several
techniques and algorithms in more technical detail than Bird et al. (2009) and
provides a variety of worked examples and code snippets.
Segaran (2007) Programming Collective Intelligence. This highly readable and
informative text includes tutorial material on machine learning techniques using
the Python language.
Additional reading
Russell and Norvig (2010) Artificial Intelligence: a modern approach, third edition.
This book is currently regarded as the definitive textbook in Artificial
Intelligence, and includes useful material on natural language processing as well
as on machine learning, which has many applications in NLP.
Mitkov (2003) The Oxford Handbook of Computational Linguistics. Edited by Ruslan
Mitkov. A collection of short articles on major topics in the field, contributed by
acknowledged experts in their respective disciplines.
Partee et al. (1990) Mathematical Methods in Linguistics. A classic text, whose
contents indicate how much the field has changed since its publication. A book
with such a title nowadays would be expected to include substantial coverage of
statistics, probability and information theory, but this text is devoted exclusively
to discrete mathematics including set theory, formal logic, algebra and automata.
These topics are particularly applicable to the content of Chapters 2 and 6.
Websites
Introductory/Reference The Internet Grammar of English is a clear and informative
introductory guide to English grammar which also serves as a tutorial in
grammatical terminology and concepts. The site is hosted by the Survey of
English Usage at University College London
(http://www.ucl.ac.uk/internet-grammar/home.htm, last visited 27th May
2013).
Hands-on corpus analysis
BNCWeb is a web-based interface to the British National Corpus hosted at Lancaster
University which supports a variety of online queries for corpus analysis
(http://bncweb.info/; last visited 27th May 2013).
The Bank of English forms part of the Collins Corpus, developed by Collins
Dictionaries and the University of Birmingham. Used as a basis for Collins
Advanced Learner’s Dictionary, grammars and various tutorial materials for
learners of English. Limited online access at
http://www.collinslanguage.com/wordbanks; (last visited 27th May 2013).
Journals and conferences
Computational Linguistics is the leading journal in this field and is freely available at
http://www.mitpressjournals.org/loi/coli (last visited 27th May 2013).
Conference Proceedings are often freely downloadable and many of these are
hosted by the ACL Anthology at http://aclweb.org/anthology-new/ (last visited
27th May 2013).
7
CO3354 Introduction to natural language processing
1.5 Software requirements
This course assumes you have access to the Natural Language Toolkit (NLTK) either
on your own computer or at your institution. The NLTK can be freely downloaded
and it is strongly recommended that you install it on your own machine: Windows,
Mac OSX and Linux distributions are available from http://nltk.org (last visited
April 10th 2013) and some distributions of Linux have it in their package/software
managers. Full instructions are available at the cited website along with details of
associated packages which should also be installed, including Python itself which is
also freely available. Once you have installed the software you should also download
the required datasets as explained in the textbook (Bird et al., 2009, p. 3).
You should check the NLTK website to determine what versions of Python are
supported. Current stable releases of NLTK are compatible with Python 2.6 and 2.7.
A version supporting Python 3 is under development and may be available for
testing by the time you read this guide (as of April 2013).
1.6 How to use the guide/structure of the course
This section gives a brief summary of each chapter. These learning outcomes are
listed at the beginning of each main chapter and assume that you have worked
through the recommended readings and activities for that chapter.
1.6.1 Chapter 2: Introducing NLP: patterns and structures in language
This chapter looks at different approaches to analysing texts, ranging from ‘shallow’
techniques that focus on individual words and phrases to ‘deeper’ methods that
produce a full representation of the grammatical structure of a sentence as a
hierarchical tree diagram. The chapter introduces two important formalisms:
regular expressions, which will play an important part throughout the course, and
context-free grammars which we return to in Chapter 6 of the subject guide.
1.6.2 Chapter 3: Getting to grips with natural language data
This chapter looks at the different kinds of data resources that can be used for
developing tools to harvest information that has been published as machine-readable
documents. In particular, we introduce the notion of a ‘corpus’ (plural corpora) – for
the purposes of this course, a computer-readable collection of text or speech. The
NLTK includes a selection of excerpts from several well-known corpora and we
provide brief descriptions of the most important of these and of the different formats
in which corpora are stored.
8
What the course does not cover
1.6.3 Chapter 4: Computational tools for text analysis
The previous chapter introduced some relatively superficial techniques for language
analysis such as concordancing and collocations. This chapter covers some
fundamental operations in text analysis:
tokenisation: breaking up a character string into words, punctuation marks and
other meaningful expressions;
stemming: removing affixes from words, e.g. mean+ing, distribut+ion;
tagging: associating each word in a text with a grammatical category or part of
speech.
1.6.4 Chapter 5: Statistically-based techniques for text analysis
Statistical and probabilistic methods are pervasive in modern computational
linguistics. These methods generally do not aim at complete understanding or
analysis of a text, but at producing reliable answers to well-defined problems such as
sentiment analysis, topic detection or recognising named entities and relations
between them in a text.
1.6.5 Chapter 6: Analysing sentences: syntax and parsing
This chapter resumes the discussion of natural language syntax that was introduced
in Chapter 2, concentrating on context-free grammar formalisms and various ways
they need to be modified and extended beyond the model that was presented in that
chapter. Formal grammars do not encode any kind of processing strategy but simply
provide a declarative specification of the well-formed sentences in a language.
Parsers are computer programs that use grammar rules to analyse sentences, and
this chapter introduces some fundamental approaches to syntactic parsing.
1.6.6 Appendices
The Appendices include:
A. A bibliography listing all works referenced in the subject guide, including
publication details and ISBNs.
B. A glossary of technical terms used in this subject guide.
C. Answers to selected activities.
D. A trace of a recursive descent parse as described in Chapter 6 of the subject guide.
E. A sample examination paper with guidelines on how to answer questions.
1.7 What the course does not cover
The field of natural language processing or computational linguistics is a large and
diverse one, and includes many topics we will not be able to address in this course.
Some of these are listed below:
9
CO3354 Introduction to natural language processing
speech recognition and synthesis
dialogue and question answering
machine translation
semantic analysis, including word meanings and logical structure
generating text or speech from non-linguistic inputs.
However, the course should provide you with a basis for investigating some of these
areas for your final year project. Some of these topics are dealt with in the later
chapters of Bird et al. (2009) and most of them are touched on by Jurafsky and
Martin (2009).
10
Chapter 2
Introducing NLP: patterns and structure in
language
Essential reading
Steven Pinker (2007), The Language Instinct, Chapter 4.
Recommended reading
Jurafsky and Martin (2009), Speech and Language Processing second edition,
Chapters/Sections 2 ‘Regular Expressions and Automata’, 5.1 ‘(Mostly) English Word
Classes’, 12.1 ‘Constituency’, 12.2 ‘Context-Free Grammars’, 12.3 ‘Some Grammar
Rules for English’.
Additional reading
The Internet Grammar of English;
http://www.ucl.ac.uk/internet-grammar/home.htm especially sections ‘Word
Classes’ and ‘Introducing Phrases’.
Partee, ter Meulen and Wall, (1990), Mathematical Methods in Linguistics,
Chapters/Sections 16.1–4, 17.1–3 (omitting 17.1.2–5, 17.2.1), 18.2, 18.6.
2.1 Learning outcomes
By the end of this chapter, and having completed the Essential reading and activities,
you should be able to:
explain the concept of finite state machines (FSMs) and their connections with
regular expressions; work through simple FSMs
write regular expressions for well-defined patterns of symbols
analyse sentences in terms of parts of speech (POS) and constituent structure,
including the use of tree diagrams
write regular and context-free grammars for small fragments of natural language
explain the concept of stemming and specify word-formation rules for given
examples.
11
CO3354 Introduction to natural language processing
2.2 Introduction
People communicate in many different ways: through speaking and listening,
making gestures, using specialised hand signals (such as when driving or directing
traffic), using sign languages for the deaf, or through various forms of text.
By text we mean words that are written or printed on a flat surface (paper, card,
street signs and so on) or displayed on a screen or electronic device in order to be
read by their intended recipient (or by whoever happens to be passing by).
This course will focus only on the last of these: we will be concerned with various
ways in which computer systems can analyse and interpret texts, and we will assume
for convenience that these texts are presented in an electronic format. This is of
course quite a reasonable assumption, given the huge amount of text we can access
via the World Wide Web and the increasing availability of electronic versions of
newspapers, novels, textbooks and indeed subject guides. This chapter introduces
some essential concepts, techniques and terminology that will be applied in the rest
of the course. Some material in this chapter is a little technical but no programming
is involved at this stage.
We will begin in section 2.3 by considering texts as strings of characters which can
be broken up into sub-strings, and introduce some techniques for informally
describing patterns of various kinds that occur in texts. Subsequently in section 2.4
we will begin to motivate the analysis of texts in terms of hierarchical structures in
which elements of various kinds can be embedded within each other, in a
comparable way to the elements that make up an HTML web document. This section
introduces some technical machinery such as: finite-state machines (FSMs), regular
expressions, regular grammars and context-free grammars.
2.3 Basic concepts
2.3.1 Tokenised text and pattern matching
One of the more basic operations that can be applied to a text is tokenising:
breaking up a stream of characters into words, punctuation marks, numbers and
other discrete items. So for example the character string
“Dr. Watson, Mr. Sherlock Holmes”, said Stamford, introducing us.
can be tokenised as in the following example, where each token is enclosed in single
quotation marks:
`"' `Dr.' `Watson' `,' `Mr.' `Sherlock' `Holmes' `"' `,'
`said' `Stamford' `,' `introducing' `us' `.'
At this level, words have not been classified into grammatical categories and we
have very little indication of syntactic structure. Still, a fair amount of information
may be obtained from relatively shallow analysis of tokenised text. For example,
suppose we want to develop a procedure for finding all personal names in a given
text. We know that personal names always start with capital letters, but that is not
enough to distinguish them from names of countries, cities, companies, racehorses
12
Basic concepts
and so on, or from capitalisation at the start of a sentence. Some additional ways to
identify personal names include:
Use of a title Dr., Mr., Mrs., Miss, Professor and so on.
A capitalised word or words followed by a comma and a number, usually below
100: this is a common way of referring to people in news reports, where the
number stands for their age – for example Pierre Vinken, 61, . . .
A capitalised word followed by a verb that usually applies to humans: said,
reported, claimed, thought, argued . . . This can over-generate in the case of
country or organisation names as in the Crown argues or Britain claimed.
We can express these more concisely as follows, where j is the disjunction symbol,
Word stands for a capitalised word and Int is an integer:
(Dr. j Professor j Mr. j Mrs. j Miss j Ms) Word
Word Word, Int
Word (said j thought j believed j claimed j argued j ...)
Learning activity
1. Write down your own examples of names that match each of the above patterns.
2. Pick a newspaper article or webpage that provides a variety of examples of people’s names. Do they
match the patterns we have encoded above? If not, see if you can devise additional rules for
recognising names and write them out in a similar format.
2.3.2 Parts of speech
A further stage in analysing text is to associate every token with a grammatical
category or part of speech (POS). A number of different POS classifications have
been developed within computational linguistics and we will see some examples in
subsequent chapters. The following is a list of categories that are often encountered
in general linguistics: you will be familiar with many of them already from learning
the grammar of English or other languages, though some terms such as Determiner
or Conjunction may be new to you.
Noun fish, book, house, pen, procrastination, language
Proper noun John, France, Barack, Goldsmiths, Python
Verb loves, hates, studies, sleeps, thinks, is, has
Adjective grumpy, sleepy, happy, bashful
Adverb slowly, quickly, now, here, there
Pronoun I, you, he, she, we, us, it, they
Preposition in, on, at, by, around, with, without
Conjunction and, but, or, unless
Determiner the, a, an, some, many, few, 100
13
CO3354 Introduction to natural language processing
Bird et al. (2009, pp. 184–5) make the standard distinction that nouns ‘generally
refer to people, places, things or concepts’ while verbs ‘describe events or actions’.
This may be helpful when one is starting to learn grammatical terminology but is
something of an over-simplification. One can easily find or construct examples
where the same concept can be expressed by a noun or a verb, or by an adjective or
an adverb. And on the other hand, there are many words that can take different
parts of speech depending on what they do in a sentence:
1. Rome fell swiftly.
2. The fall of Rome was swift.
3. The enemy completely destroyed the city.
4. The enemy’s destruction of the city was complete.
5. John likes to fish on the river bank.
6. John caught a fish.
Additionally, some types of verbs do not correspond to any particular action but
serve a purely grammatical function: these include the auxiliary verbs such as did,
shall and so on. So in summary, we can often only assign a part of speech to a word
depending on its function in context rather than how it relates to real things or
events in the world.
Learning activity
Identify parts of speech in these examples:
1. The cat sat on the mat.
2. John sat on the chair.
3. The dog saw the rabbit.
4. Jack and Jill went up the hill.
5. The owl and the pussycat went to sea.
6. The train travelled slowly.
2.3.3 Constituent structure
You will have noticed several recurring patterns in the above examples: Det Noun,
Prep Det Noun and so on. You may also have noticed that some types of phrase can
occur in similar contexts: (John j the cat) sat, a Proper Noun or a sequence Det Noun
can come before a Verb. Some of these possibilities can be captured using the
pattern-matching notation introduced above, for example:
(((the j a)(cat j dog))(John j Jack j Susan))(barked j slept)
This will match any sequence which ends in a verb barked or slept preceded by
either a Determiner a or the followed by a Noun cat or dog or a proper name John,
Jack or Susan.
Patterns that have similar distributions (meaning that they can occur in similar
contexts) are standardly identified by phrasal categories such as Noun Phrase or Verb
14
A closer look at syntax
Phrase. A common way to represent information about constituent structure is by
means of production rules of the form X ! A;B;C : : :. Using rules of this form,
grammatical sentences can be broken down into constituent phrases consisting of
various combinations of POS:
Sentence!Noun Phrase, Verb Phrase
Noun Phrase!Determiner, Noun (Example: the, dog)
Noun Phrase!Proper Noun (Example: Jack)
Noun Phrase!Noun Phrase, Conj, Noun Phrase (Examples: Jack and Jill, the owl and the pussycat)
Verb Phrase!Verb, Noun Phrase (Example: saw the rabbit)
Verb Phrase!Verb, Preposition, Noun Phrase (Examples: went up the hill, sat on the mat)
Learning activity
Read through the recommended sections of the UCL ‘Internet Grammar of English’. Write production rules
that cover some of the examples in these sections.
2.4 A closer look at syntax
This section aims to motivate the idea that texts can be analysed as hierarchical
structures rather than ‘flat’ sequences whose elements are organised in various
patterns. The Essential reading for this chapter by Steven Pinker gives a concise and
accessible introduction to some fundamental distinctions we will make in this
section, from the point of view of Chomskyan linguistics (compare Chomsky,
1957/2002). Chomsky and his followers argue that some components of our
knowledge of language are innate, and Pinker (2007, chapter 4) sketches some
arguments in support of this claim. This position is considered to be contentious by
many linguists and we will not address it in this course. However, Pinker’s chapter
provides a useful introduction to syntactic analysis and clearly distinguishes between
two formal techniques for modelling grammatical knowledge, which underlie
regular and context-free grammars respectively (these terms will be explained as
we go along).
Learning activity
If you have access to it, read through the recommended chapter by Pinker and make notes, and have it to
hand while working through the remainder of this section.
Pinker notes that language makes ‘infinite use of finite means’, in Humboldt’s
phrase1. That is, there is no principled upper limit to the length of a grammatical
sentence: we can always add another phrase, even if it’s a banal one like ‘one could
say that’, ‘and that’s a fact’ or ‘and you can tell that to the Marines’. A large
1‘Sie [die Sprache] muss daher von endlichen Mitteln einen unendlichen Gebrauch machen’ (von Humboldt,
1836, p. 122).
15
CO3354 Introduction to natural language processing
proportion, perhaps most of the sentences we read, hear or speak every day may be
entirely novel, at least to us. Consequently, knowledge of a language seems to
consist in knowing rules that specify what sentences belong to the language, rather
than memorising long lists of sentences to be produced on appropriate occasions.
Pinker considers two different formal systems for generating or recognising
sentences in English:
‘wordchain’ devices, equivalent to finite state machines. These devices
incorporate three distinct operations: sequence, selection and iteration.
Phrase structure grammars, which include the additional operation of recursion.
Note that Pinker deliberately uses the more descriptive expression ‘wordchain’ as he
is concerned to avoid the use of forbidding technical terminology. In what follows
we will stick to the standard term finite-state machine which you are more likely to
find in textbooks. You may also encounter the terms finite-state automaton or just
finite automaton.
2.4.1 Operation of a finite-state machine
A wordchain device or finite-state machine (FSM) can be seen as a set of lists of
symbols (such as words or fixed phrases) and rules for going from list to list. A
simple example:
Word lists
1. The, a, one
2. Cat, dog, fish
3. Barked, slept, swam
Rules
It is important to keep in mind that FSMs are neutral between accepting and
generating strings. That is to say, one way to operate a FSM is to read a string, one
symbol at a time, and determine whether the symbol is found in the list at the
current state of the machine. If it is, we advance to the next state and read the next
symbol. Alternatively, this FSM could be used to generate strings by picking one
word from each list in sequence. Some possible matching strings are:
The dog swam
A cat barked
A fish slept
. . .
A more complex example:
1. John/Mary/Fred OR
1a. the/a/one
1b. cat/dog/fish
16
A closer look at syntax
2. (optional): and/or GO TO 1
3. slept/barked/swam OR
3a. sat/walked
3b. on
3c. a/the
3d. mat/hill
4. (optional) and/or GO TO 3
5. (optional) and/or/but GO TO 1.
This formulation involves the basic operations of sequence, selection and iteration as
follows:
SEQUENCE
Moving from list to list in numerical order: 1, 2, 3 . . .
SELECTION
Choosing an item from a list, for example cat, dog or fish; choosing between lists.
ITERATION
Repeating particular sequences, for example:
John and Mary or a fish (repeats step 1.)
The cat barked but Fred walked on the hill. (Repeats steps 1–5, omitting step 4.)
Learning activity
1. Find the shortest sentence generated or accepted by the above FSM.
2. Write out four sentences between six and 20 words long which are accepted by the FSM.
2.4.2 Representing finite-state machines
There are various conventional ways of representing a non-deterministic FSM in
terms of a number of states and the permissible transitions between states. In our
informal exposition above, the numbered steps represent states and each symbol or
word in a list counts as a possible transition to the next state. Pinker adopts a
graphical convention where states are depicted as nodes in a graph and transitions
are directed, labelled arcs between the nodes; see also Partee et al. (1990, p. 457
and following). Alternatively, the states and transitions can be shown in tabular form
as in Table 2.1 where q1 is the initial state and q4 the final state:
17
CO3354 Introduction to natural language processing
q1 john q2
q1 mary q2
q1 the q1a
q1 a q1a
q1a cat q2
q1a dog q2
q2 slept q3
q2 barked q3
q2 swam q3
q3 and q1
q3 or q1
q3 . q4
Table 2.1: A finite-state machine represented as a state-transition table.
2.4.3 Declarative alternatives to finite-state machines
The FSMs shown above combine a formal specification of a language with a
processing strategy. It is often convenient to separate the two and define the
language using expressions from a declarative formalism which can be manipulated
using various different algorithms. This section considers two such formalisms:
regular expressions and regular grammars.
Regular expressions (REs) provide a simple but powerful means of identifying
patterns in text and are widely used in various applications of computer science. REs
are based on three fundamental concepts which as we have seen are characteristic of
finite-state machines:
sequence – to do with the order in which items occur: may include a wildcard
character which is written as the period or full stop ‘.’ and may be replaced by
any character.
selection – specifying a choice between alternative items or sequences, indicated by
the ‘j’ operator
iteration – repetition of items or sequences, indicated by the ‘*’ operator, meaning
zero or more occurrences of whatever precedes the star.
Some simple examples:
a* matches sequences of zero or more a’s: a, aaaa, aaaaaaaaaaa and so on. A
sequence of zero elements is known as the ‘empty string’ and conventionally
denoted by the Greek letter epsilon or .
aa* sequences of one or more a’s
ab* sequences of one a followed by zero or more b’s: a, ab, abbbb, . . .
(ab)* sequences of zero or more pairs ab: , ab, abab, ababab . . .
(ab)j(ba) ab or ba
((ab)j(ba))* possibly empty sequences of ab and ba pairs: , ab, abab, baab,
bababa, abba . . . Note that parentheses operate in the usual manner as in
mathematical or logical expressions, to denote the scope of operators.
b.*a all strings that start with b and end with a: ba, bbbaaaa, bcccccccca . . .
18
A closer look at syntax
Programming languages such as Java, Perl and Python implement extensions of REs
with operators which are mostly redundant in that they can be reduced to
combinations of the above operations, but can make programs much more compact
and readable, including:
+ – one or more of the previous item
? – the previous item is optional
[A-Z], [0-9] – this expression matches one of a range of characters
ˆabc – matches pattern abc at the start of a string
abc$ – matches pattern abc at the end of a string.
See also Bird et al. (2009, Table 3.3) and the other recommended readings on this
topic.
Here are some examples of our suggested ways of recognising personal names coded
as regular expressions. These are intended to be applied to tokenised text and every
sequence enclosed by angled brackets < : : : > stands for an individual token. In
Examples 1, 3 and 4 below, the material within parentheses represents the target
string and sequences outside parentheses provide the context.
1. <Mrs?>(<.+>) matches ‘Mr’ or ‘Mrs’ followed by any string. The first token
consists of the sequence Mr followed optionally by the character s. The second
consists of a sequence of one or more characters: any character may occur in
place of the wildcard ‘.’.
2. <[A-Z][a-z]+>+ matches any sequence of one or more capitalised words.
3. (<[A-Z][a-z]+>+)<,><[0-9]+> matches capitalised word(s) followed by a
comma and a number (age).
4. (<[A-Z][a-z]+>+)<saidjreportedjclaimed>.
Learning activity
1. Write a regular expression for all strings consisting of an odd number of a’s followed by an even
number of b’s.
2. Write a regular expression for all sequences of a’s and b’s of length 3.
3. Write a regular expression for all strings that contain abba somewhere within them.
As you have probably observed, the pattern-matching notation we used in section
2.3 employed a subset of the RE syntax, and we could in principle use regular
expressions to encode simple grammars as presented in that section. For example:
( (JohnjMaryjFred) j ( (theja)(catjdogjfish) )
(barked j slept j swam)
((and j or) (barked j slept j swam))*
matches sentences like:
1. John slept
2. The cat barked or swam
3. Mary swam and barked or slept
19
CO3354 Introduction to natural language processing
4. . . .
It can be seen that even conceptually simple REs can rapidly become almost
unreadable. A more manageable formalism is a regular grammar, made up of
production rules or rewrite rules of the kind you have seen in the previous section:
S!John j Mary j Fred VP
S!the j a S1
S1!cat j dog j fish VP
VP!barked j slept j swam VP1
VP1!and j or VP
VP1!
A sequence of words forms a grammatical sentence according to a grammar of this
type if one can draw a tree diagram like Figure 2.1 such that:
1. The root node is S or Sentence.
2. For every node that matches the left hand side (LHS) of a grammar rule, one can
draw a subtree with the items on the right hand side (RHS) as daughter nodes.
3. When no more grammar rules can be applied, every leaf node of the tree
matches a word in the language or the empty string .
S
the
S1
cat
VP
barked
VP1
or
VP
swam
VP1

Figure 2.1: A right-branching tree.
Symbols which only occur on the right-hand side of rules, and so can only appear as
leaf-nodes in a tree, are known as terminal symbols. Regular grammars have the
restriction that when non-terminal symbols appear on the RHS they must either
always be the rightmost symbol, or always the leftmost. These classes of grammars
are known as right-linear and left-linear respectively. A right-linear grammar will
always result in a right-branching tree as in the above example.
20
A closer look at syntax
Learning activity
Draw tree diagrams according to the above grammar for the sentences:
1. The dog slept.
2. Mary swam and barked or slept.
2.4.4 Limitations of finite-state methods – introducing context-free grammars
Pinker (2007, p.86) gives an example of a ‘wordchain device’ or FSM which is
intended to show the limitations of finite-state methods for handling natural
language. The procedure is apparently designed to deal with complex sentences
including constructions like If. . . then. . . and Either. . . or . . . . If we look at a few
possible matching strings, we see clearly that some are grammatical sentences but
others are nonsensical. (Following a standard convention in linguistics, the
unacceptable cases are marked with an asterisk ‘*’.)
Either a happy girl eats ice cream or the boy eats hot dogs.
*Either a happy girl eats ice cream then one dog eats candy.
If a girl eats ice cream then the boy eats candy.
*If a girl eats ice cream or the boy eats candy.
Learning activity
1. Write a regular grammar that is equivalent to the FSM in Pinker (2007, p. 86).
2. Convince yourself that it allows you to draw well-formed trees for the ungrammatical examples above.
3. What characteristic of the grammar prevents it from ruling out the ill-formed examples?
See ‘Answers to Activities’ in Appendix C (p. 98) for further discussion.
In order to handle these kinds of sentences correctly we need to add new kinds of
rewrite rules, going beyond the class of right- or left-linear grammars:
1. To match pairs of words like if . . . then, either . . . or, we need rules where a
non-terminal symbol on the RHS can have additional material on both sides as
in the first two rules below.
2. In order to allow for indefinite nesting – if either John will come if Mary does, or
. . . we need rules where the same symbol can occur on both sides of the arrow.
This is known as self-embedding or centre-embedding.
Note that centre-embedding is an instance of recursion in grammar; right-linear
grammars may also include recursive rules but they can always be processed
iteratively rather than recursively (Jurafsky and Martin, 2009, p. 447).
21
CO3354 Introduction to natural language processing
S!Either S or S
S!If S then S
S!NP VP
NP!Det N
Det!a j the j 
N!girl j boy j dog j cat j burgers j candy j cream j cake
VP!V NP
VP!V PP
PP!P NP
V!eats j likes j sat
P!on
The above grammar handles these cases correctly as well as simple sentences like
The cat sat on the mat:
S
NP
Det
the
N
cat
VP
V
sat
PP
P
on
NP
Det
the
N
mat
Figure 2.2: Tree diagram for The cat sat on the mat.
It is also acceptable to represent trees using labelled bracketed strings as in the
example below:
(S
(NP (Det the ) (N cat ))
(VP (V sat )
(PP
(P on )
(NP
(Det the ) (N mat ) )
)
)
)
22
A closer look at syntax
Figure 2.3 is an example of self-embedding.
S
If S
NP
Det
the
N
cat
VP
V
likes
NP
Det

N
cream
then S
NP
Det
the
N
boy
VP
V
eats
NP
Det

N
burgers
Figure 2.3: Tree diagram with self-embedding.
Learning activity
1. Trace through the grammar rules and satisfy yourself that Figure 2.3 represents the structure of the
sentence If the cat likes cream then the boy eats burgers according to the grammar.
2. What is the shortest sentence generated by the above grammar?
3. Using the above grammar, draw complete tree diagrams for:
(a) If the girl likes cake then either the boy eats burgers or the boy eats candy.
(b) If either the boy likes cake or the girl likes burgers then the dog eats candy.
4. Think of ways to modify the grammar to generate more natural-sounding sentences.
2.4.5 Looking ahead: some further uses of regular expressions
In this chapter we have so far looked at finite-state formalisms as techniques for
generating or recognising short phrases as well as whole sentences, and found them
to be wanting. Many current applications in language technology do not, in fact,
require complete analysis of sentences but proceed by looking for patterns of interest
within a text and discarding what does not match these patterns. Finite-state
methods are often quite adequate for these applications and you will see many uses
for regular expressions in later chapters of this guide. Some examples we will look at
in more detail in later chapters are:
stemming: extracting the ‘base form’ of a word as informally presented in
section 2.5 of this chapter
tagging: automatically assigning POS or other forms of mark-up to elements in a
text
chunking: grouping together a sequence of words as a phrase
information extraction: identifying chunks that denote meaningful entities,
events or other items of interest.
23
CO3354 Introduction to natural language processing
2.4.6 Looking ahead: grammars and parsing
The pseudocode and graphical representations of wordchains (FSMs) combine a
specification of the well-formed sentences in a language fragment with a processing
strategy. It is important to keep in mind that formal grammars made up of a series of
production rules do not encode a processing strategy. As stated above, a grammar is
a declarative specification of the strings that make up a language while parsers use a
variety of algorithms to apply the grammar rules. We will look at some of these
parsing strategies in Chapter 6 of this subject guide.
2.5 Word structure
Words combine in different orders to form sentences and phrases; they also have
internal structure. Nouns in English may have different endings according to
whether they are singular (a box) or plural (some boxes) while in some languages
this information may be expressed at the start of the word, for example Swahili ziwa
(‘lake’) vs maziwa (‘lakes’). In English, endings of verbs can indicate person, number,
tense and mood2, while other languages may make different dictinctions. Nouns and
verbs are sometimes classified as regular or irregular according to whether their
inflected forms can be derived by following simple rules. Table 2.2 shows examples
of some common past tense forms in English.
Present Past
become became
come came
mistake mistook
misunderstand misunderstood
ring rang
sell sold
shake shook
sing sang
sink sank
stand stood
take took
tell told
travel travelled
understand understood
withstand withstood
Table 2.2: Past tense forms (1).
The subfield of linguistics known as morphology is concerned with the structure of
words and is concerned, among other things, with formulating rules for deriving
different forms of a word according to its grammatical role. Here are some rules
which appear to cover the examples in the table:
2See the Internet Grammar of English at http://www.ucl.ac.uk/internet-grammar/verbs/verbs.htm (last
visited 27th May 2013) for explanations of these terms.
24
A brief history of natural language processing
Some rules for past-tense formation
-come ! -came
-take ! -took
-ing ! -ang
-ink ! -ank
-ell ! -old
-and ! -ood
-el ! -elled
Some of these rules could be made more general: we could combine the -ing and
-ink rules to a single rule, -in ! -an . On the other hand, some rules which work for
these particular examples would fail if applied to a wider range of data: we have
come ! came, become ! became but not welcome ! *welcame. This is an example of
rules overfitting the data.
Learning activity
Modify the above rules so that they will account for the past tense forms in Table 2.3 as well as in Table 2.2.
Present Past
bake baked
command commanded
bring brought
sling slung
smell smelt
think thought
wake woke
Table 2.3: Past tense forms (2).
A natural language application such as a machine translation system will typically
include a database of words or lexicon along with rules for deriving word endings:
for example, a translation from English into Dutch might handle the word brought as
follows:
1. Find the stem of brought and interpret the inflection: bring+past
2. Find the Dutch equivalent of bring: brengen
3. Find the past tense of brengen: bracht
The process of removing affixes from words to derive the basic form is called
stemming. We will look at some tools for doing this in Chapter 4, and you will also
have the opportunity to encode your own rules as regular expressions.
2.6 A brief history of natural language processing
The field of natural language processing or computational linguistics builds on
techniques and insights from a number of different disciplines, principally
25
CO3354 Introduction to natural language processing
theoretical linguistics and computer science but with some input from mathematical
logic and psychology.
The notions of a finite-state machine and context-free grammar (CFG) were first
introduced to linguistics by Chomsky (1957; see Pullum (2011) for a somewhat
critical reappraisal). Chomsky argued that both formalisms were inadequate for
modelling natural language and proposed an additional operation of
transformations, which could essentially permute the output string of a CFG in
various ways. Chomsky’s work introduced a methodology which was to dominate
theoretical linguistics for the next couple of decades: linguists concentrated on
postulating formal rules of grammar which were tested against their own intuitions
or those of native speakers of other languages, rather than seeking to induce rules
from large collections of data. Part of the rationale for this was that native speakers
of a language are able to recognise whether a sequence of words makes up an
acceptable sentence in their language, even if they have never encountered those
words in that particular order before. Prior to what was to become known as the
Chomskyan revolution, corpus-based approaches had been the norm in general
linguistics. This tradition was overshadowed for a time by so-called ‘generative’
linguistics, but corpus-based research continued in some quarters until its resurgence
in the 1980s, including the development of the first machine-readable corpus by the
Jesuit priest Fr Robert Busa. Busa developed a 10 million-word corpus of medieval
philosophy on punch-cards, with the support of Thomas Watson of IBM (McEnery
and Wilson, 2001, pp. 20–21).
Work in formal grammar tended to assume a ‘backbone’ of context-free rules,
augmented with various mechanisms to handle data that appeared to go beyond the
context-free model; some important developments were Generalised Phrase
Structure Grammar (Gazdar et al., 1985) and Head-driven Phrase Structure
Grammar (Pollard and Sag, 1994). We will see examples of these extra mechanisms
in Chapter 6.
Early work on automated language processing was essentially procedural in its
methodology, working with a type of finite-state machine called transition
networks which were extended as augmented transition networks to cope with
various linguistic constructions (Woods, 1970). Later work based on declarative
grammar formalisms employed techniques including deductive parsing (Pereira and
Warren, 1983) and unification (Kay, 1984). The former adopts techniques from the
AI field of automated reasoning: the core idea is that parsing a sentence can be seen
as constructing a logical proof that a particular sequence of words forms a proper
sentence according to a given set of grammar rules. Unification grammars treat
linguistic objects as sets of attributes or features with a finite range of values, and
grammar rules specify that particular items in a sentence must have the same or
compatible values for certain features. For example, the subject and main verb of a
sentence should have the same value for the number feature. We will consider
detailed examples in Chapter 6.
Meanwhile, substantial progress was made in lower-level tasks such as speech
recognition and morphological analysis using probabilistic techniques and
finite-state models. During the late 1990s these techniques were extended to cover
tasks such as parsing, part-of-speech tagging and reference resolution (recognising
whether or not different expressions in a document referred to the same person or
entity). These developments were driven by a number of factors: the continuing
increase in the processing speed and memory capacity of computers; the availability
of massive amounts of spoken and written material, both in unstructured form on
the world wide web and with various types of annotation in corpora such as the
26
Sample examination questions
Penn Treebank3 or the British National Corpus4, and events such as the Message
Understanding Conferences5 which were initially sponsored by the US Department
of Defense to measure and foster progress in extracting information from
unstructured text.
Much work since around the year 2000 has involved the use of machine learning
techniques such as Bayesian models and maximum entropy (see Chapter 5). This
has involved using annotated corpora to train systems to segment and annotate texts
according to various morphological, syntactic or semantic criteria. These techniques
have been systematically applied to particular tasks such as parsing, word sense
disambiguation, question answering and summarisation.
2.7 Summary
1. This chapter has characterised the subject matter of the course as being
concerned with various ways of using computer programs to analyse text, by
which we mean words, numbers, punctuation and other meaningful symbols
that are printed on paper or some other flat surface, or displayed on a screen.
2. Some fundamental operations in text analysis include tokenisation, which
involves extracting these meaningful elements from a stream of electronic
characters and discarding white space, line feed characters and other material
which has no explicit meaning for a human reader, and using regular
expressions to identify patterns in a text.
3. Regular expressions are composed of the three basic operations of sequence,
selection and iteration, and have many applications in computational linguistics
and computer science at large. A finite-state machine is a process whose
operations can be specified by means of regular expressions. A regular grammar
is a set of production rules or rewrite rules that defines the sentences that make
up a language, and any language defined by a regular grammar can be
processed by a finite state machine or described using a regular expression.
4. A complete syntactic analysis of natural language sentences is generally held to
require the additional operation of centre-embedded recursion, which is beyond
the power of finite-state methods. Recursion is formally encoded in context-free
grammars.
5. Not only do words combine in various patterns and structures to form sentences;
they also have internal structure which can be described to an extent using rules
for regular and irregular forms.
6. The current state of NLP or computational linguistics builds on research results
and concepts from many different fields, and we have sketched some of the
highlights in a very short history of the discipline.
2.8 Sample examination questions
You can expect a list of RE operators to be included as an appendix in the
examination paper.
3http://www.cis.upenn.edu/treebank/; last visited 27th May 2013
4http://www.natcorp.ox.ac.uk/; last visited 27th May 2013
5http://www.itl.nist.gov/iaui/894.02/related projects/muc/; last visited 27th May 2013
27
CO3354 Introduction to natural language processing
1. S ! NP VP
NP ! Det N
NP ! PN
VP ! V
VP ! V NP
VP ! V NP PP
VP ! VP Adv
PP ! P NP
Det ! the j a
N ! waiter j chairs j tables j hotel j manager
PN ! Oscar j Paris
V ! died j put j saw j called
Adv ! suddenly j quickly j slowly
P ! in j on
(a) Using the grammar rules above, draw syntax trees for:
i. Oscar died suddenly.
ii. The waiter put the chairs on the tables.
iii. Oscar called the waiter.
(b) Modify the grammar so that it generates the unstarred sentences below as
well as (i–iii) above but not the starred ones. Explain the reasons for your
modifications.
i. Oscar died in Paris.
ii. Oscar died in a hotel in Paris.
iii. The waiter came to the table when Oscar called him.
iv. When Oscar called him the waiter came to the table.
v. * Oscar put
vi. * The waiter saw on the tables
vii. * The waiter put in the chairs
viii. * The waiter put the chairs
ix. * Oscar died the table
x. * When Oscar called him when the waiter came to the table.
2. Write a regular expression that will identify male and female names in context,
in an English-language text. Discuss ways in which this might over- or
under-generate.
3. Explain the difference between regular and context-free grammars and discuss
the claim that natural language grammars need at least context-free power.
4. (a) Write a regular grammar which generates the following sentences:
i. This is the kid that my father bought.
ii. This is the cat that killed the kid that my father bought.
iii. This is the dog that worried the cat that killed the kid that my father
bought.
iv. This is the stick that beat the dog that worried the cat that killed the kid
that my father bought.
(Brewer’s Dictionary of Phrase and Fable, 1896)
(b) Write out three more sentences generated by your grammar.
28
Chapter 3
Getting to grips with natural language data
Essential reading
Bird et al. (2009) Natural Language Processing with Python Chapters 1 and 2
particularly: 1.1–1.3, 2.1–2.2, 2.5.
Recommended reading
McEnery (2003) ‘Corpus Linguistics’ in Mitkov (2003) is a succinct overview of the
topic from one of the leading scholars in the field.
Additional reading
McEnery and Wilson (2001) Corpus Linguistics is an established undergraduate
textbook; Chapters 2 and 3 are especially relevant for this topic.
McEnery and Hardie (2011) Corpus Linguistics: Method, Theory and Practice. Chapter
3 addresses the important topic of research ethics for corpus linguistics, which is
often neglected in technical or academic presentations of the subject.
3.1 Learning outcomes
By the end of this chapter, and having completed the Essential reading and activities,
you should be able to:
explain what is meant by a corpus in the context of natural language processing,
and describe some different types, structures and uses of corpora
describe the characteristics of some well-known corpora and other language
resources such as the Brown corpus, Penn treebank, Project Gutenberg and
WordNet
Use online interfaces and other software tools to do some basic corpus analysis,
including concordancing and finding collocations
locate and format raw text documents and analyse them using corpus tools.
3.2 Using the Natural Language Toolkit
As stated in Chapter 1, this subject guide is not intended as a stand-alone tutorial in
using the NLTK or the Python language. You are advised to read through the
recommended sections of Bird et al. (2009) and work through the exercises marked
29
CO3354 Introduction to natural language processing
Your turn. You may also find it useful to attempt some of the exercises provided at
the end of each chapter.
From this chapter onwards you will be running Python sessions and using the NLTK.
You should get into the habit of starting sessions with the following commands:
>>> from __future__ import division
>>> import nltk, re, pprint
One of the features that makes the Python language suitable for natural language
applications is the very flexible treatment of data structures such as lists, strings and
sequences. You should be familiar with these structures from previous programming
courses, but should ensure you understand the way they are handled in Python. For
this chapter, only lists are relevant and you should study Bird et al. (2009, section
1.2) before trying any of the learning activities in this chapter.
3.3 Corpora and other data resources
As explained in the previous chapter, much natural language processing relies on
large collections of linguistic data known as corpora (plural of corpus). A corpus can
be simply defined as no more than a collection of language data, composed of
written texts, transcriptions of speech or a combination of recorded speech and
transcriptions.
Corpora fall into three broad categories (McEnery, 2003, p.450):
Monolingual corpora consist, as the name suggests, of data from a single
language.
Comparable corpora include a range of monolingual corpora in different
languages, preferably with a similar level of balance and representativeness, and
can be used for contrastive studies of those languages.
Parallel corpora include original texts in one language with translations of those
texts in one or more different languages. Parallel corpora can be used to train
statistical translation systems.
A corpus is generally expected to have additional characteristics: corpora are usually
constructed so as to be balanced and representative of a particular domain (McEnery
and Wilson, 2001, pp. 29–30). (Sometimes the term is used more loosely to cover
any large collection of language data which need not have been compiled
systematically, as in the phrase ‘the web as corpus’.) Sampling theory is a branch of
statistics that deals with questions such as: how many respondents are needed in an
opinion poll for the results to be considered to represent public opinion at large?
Similar considerations arise in corpus linguistics. This is particularly important if a
corpus is to be used for quantitative analysis of the kind described in Chapter 5: if
the corpus data is skewed or unrepresentative then results of the analysis may not be
reliable. These considerations may be less important if the corpus is collected for the
literary or historical interest of the documents that make it up, as is the case with
Project Gutenberg for example.
For example, Bird et al. (2009, pp. 407–412) refer to the TIMIT corpus, an annotated
speech corpus developed by Texas Instruments and MIT. To ensure
representativeness, it was designed to include a wide coverage of dialect variations.
Corpus builders need to exercise expert judgment in deciding on the sampling frame,
30
Some uses of corpora
or ‘the entire population of texts from which we will take our samples’ (McEnery and
Wilson, 2001, p. 78) and calculating the size of the corpus that is necessary for it to
be maximally representative. The sampling frame may, for example, be bibliographic,
based on some comprehensive index or the holdings of a particular library, or
demographic, selecting informants on the basis of various social categories as is often
done in public opinion research.
Corpora have tended to be of a finite size and to remain fixed once they have been
compiled. There are also what is known as monitor corpora which are continually
updated with new material. This is particularly useful for compilers of dictionaries
who need to be able to track new words entering the language and the changing or
declining use of old ones. An example of a monitor corpus is the COBUILD Bank of
EnglishTM, which had around 300 million words when it was referred to by McEnery
(2003) and has since more than doubled in size to 650 million words.
A further distinction is between corpora consisting solely of the original or ‘raw’ text
and those that have been marked up with various annotations. One common
technique is standoff annotation where the mark-up is stored in a different file from
the original text (McEnery and Wilson, 2001, p.38); (Bird et al., 2009, p.415).
Finally, corpora can be further classifed according to their structure:
Isolated – an unorganised collection of individual texts such as the Gutenberg
online collection of literary works.
Categorised – texts are organised by categories such as genre; an example is the
Brown corpus described below.
Overlapping – some categories overlap. A news corpus such as Reuters may contain
stories which cover both politics and sport, for example.
Temporal – texts indicate language use over time. Examples are the Inaugural
corpus of all inaugural speeches by US Presidents, and the Helsinki Diachronic
corpus of about 1.6 million words of English dating from the early 9th century
CE to 1710.
Some examples of corpora, which will be described in more detail later in the
chapter, are:
Brown Developed at Brown University in the early 1960s.
BNC British National Corpus, created and managed by BNC consortium which
includes Oxford and Lancaster universities, dictionary publishers OUP,
Longmans and Chambers, and the British Library.
COBUILD (Bank of English) The Bank of EnglishTMforms part of the Collins
Corpus, developed by Collins Dictionaries and the University of Birmingham,
and contains 650 million words.
Gutenberg An archive of free electronic books in various formats hosted at
http://www.gutenberg.org/
Penn Treebank A corpus of reports from the Wall Street Journal and other sources
in various different formats.
3.4 Some uses of corpora
McEnery and Wilson (2001, Chapter 4) discuss a variety of uses for corpora, some of
which are briefly discussed below.
31
CO3354 Introduction to natural language processing
3.4.1 Lexicography
Modern dictionaries such as Chambers, Collins and Longmans now rely heavily on
corpus data in order to classify and inventorise the various ways words can be used
in contemporary English as well as any ways these uses may have changed. For
example, a lexicographer who wishes to determine whether the words scapegoat,
thermostat or leverage can be used as verbs can enter the appropriate search string
and be presented with examples such as the following (from the BNC):
Scapegoating an individual prevents the debate and delays community
understanding.
The measuring cell is immersed in a vat of liquid, usually benzene or xylene which
can be thermostatted at temperatures between 273 and 400 K.
These one-time costs once met could be leveraged over much more business activity
around the globe than we then enjoyed.
McEnery and Wilson (2001, p. 107) discuss a case where they claim that two
well-known dictionaries had ‘got it wrong’ by listing quake as a solely intransitive
verb, while examples in a transitive construction can in fact be found through a
corpus search:
These sudden movements quake the Earth. (BNC)
It is perhaps debatable whether the dictionaries in question were ‘wrong’ to
disregard examples like this, or the compilers may have judged this to be an
idiosyncratic usage which did not merit being included in a work of reference with
the status of standard usage.
3.4.2 Grammar and syntax
Large-scale grammars for pedagogic and reference use such as the Comprehensive
Grammar of the English Language (Quirk et al., 1985) or the Cambridge Grammar of
the English Language (Huddleston and Pullum, 2002) use corpora among their
sources of information along with results of linguistic research and the compilers’
own subjective intuitions as competent speakers of the language, although this has
tended to involve qualitative rather than quantitative analysis. Recent advances in
computational power and developments in parsed corpora and tools to analyse them
have made it possible for researchers to carry out quantitative studies of various
kinds of grammatical frequency, such as the relative frequency of different clause
types in English. Other studies use corpora to test predictions made by formal
grammars that have been developed within the generative school of linguistics. The
COBUILD project which provided the resources for Collins English dictionaries has
also resulted in a series of small handbooks covering various kinds of grammatical
construction, which are useful both for advanced language learners and for linguists
in search of examples.
3.4.3 Stylistics: variation across authors, periods, genres and channels of
communication
The notion of style in communication is that people generally have a choice between
different ways of expressing themselves and not only do individuals tend to make
32
Corpora
similar choices each time they communicate, but their particular choices may be
more characteristic of particular genres (romantic fiction, financial news, court
reports and so on), time periods and channels of communication. By channels we
mean distinctions such as written text compared with spoken discourse, both of
which can be further subdivided: people will make different choices when
composing emails, sending text messages or (rarely) writing a letter by hand. We
probably also adopt different styles when talking face-to-face and on the telephone.
Literary scholars as well as law enforcement and intelligence agencies may have an
interest in identifying the author of a document from internal evidence. There have
been various famous and less well-known instances of controversial attribution of
authorship: for example, various figures have been put forward as the authors of
Shakespeare’s plays.
3.4.4 Training and evaluation
In addition to the applications listed above, corpora are routinely used in linguistic
research for training and testing machine-learning systems for specific tasks in text
analytics such as:
detecting the topic of a document
analysing the sentiments expressed for or against some product or policy
identifying individuals described in a text, relations between them and events
they participate in
statistical parsing
statistical machine translation.
For example, the Brown corpus and the WSJ corpus are typically used for evaluating
text segmentation among other text processing tasks (Mikheev, 2003, p. 203).
These topics will be covered in more detail in Chapter 5 of this subject guide, where
you will be introduced to various machine-learning techniques. These will all be
types of supervised learning, where a system is trained on ‘corpora containing the
correct label for each input’ (Bird et al., 2009, p. 222), as opposed to unsupervised
learning where the system is designed to detect patterns in the input without any
feedback. This normally means that the corpus has been marked up by human
annotators. Standard practice is to divide a corpus into training and test sets; the
test set is considered a gold standard for comparing the accuracy of trained learning
systems with that of the annotators. Of course humans are fallible, and it is good
practice to use multiple annotators for at least a sample of the corpus and report the
level of inter-annotator agreement that was achieved. This will set an upper bound
for the performance that can be expected from the system, as it seems meaningless
to say that a computer program can achieve 100 per cent accuracy on tasks where
human annotators disagree (see Jurafsky and Martin, 2009, p. 189).
3.5 Corpora
This section provides brief descriptions of various corpora, some of which are
distributed in full or in part with the NLTK and others can be accessed online.
33
CO3354 Introduction to natural language processing
3.5.1 Brown corpus
This was one of the first ‘large-scale’ machine readable corpora, though it may seem
rather small by today’s standards, consisting of one million words. It was developed
at Brown University from the early 1960s and took around two decades to complete.
It was intended as a ‘standard corpus of present-day edited American English’ and is
caterorised by genre under headings such as:
News Chicago Tribune: Society Reportage
Editorial Christian Science Monitor: Editorials
Reviews Time Magazine: Reviews
Government US Office of Civil Defense: The Family Fallout Shelter
Science Fiction Heinlein: Stranger in a Strange Land
Humour Thurber: The future, if any, of comedy.
The Brown corpus is provided with the NLTK in tagged and untagged versions and
can be accessed using the various methods listed by Bird et al. (2009, Table 2.3,
p. 50).
