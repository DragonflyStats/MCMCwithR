
NLP Class: Intro



Introduction and Course Overview

First Two Weeks

Levenshtein distance (Minimum Edit Distance)

Principle Of Maximum Entropy

Sentiment analysis




--------------------------------------------------------------------------------



Introduction and Course Overview


We will also introduce the underlying theory from probability, statistics, and machine learning that are crucial for the field, and cover fundamental algorithms like n-gram language modeling, naive bayes and maxent classifiers, sequence models like Hidden Markov Models, probabilistic dependency and constituent parsing, and vector-space models of meaning.


The course covers a broad range of topics in natural language processing, including word and sentence tokenization, text classification and sentiment analysis, spelling correction, information extraction, parsing, meaning extraction, and question answering, We will also introduce the underlying theory from probability, statistics, and machine learning that are crucial for the field, and cover fundamental algorithms like n-gram language modeling, naive bayes and maxent classifiers, sequence models like Hidden Markov Models, probabilistic dependency and constituent parsing, and vector-space models of meaning.



--------------------------------------------------------------------------------


First Two Weeks


The following topics will be covered in the first two weeks:
1.
Introduction and Overview:

2.
Basic Text Processing: J+M Chapters 2.1, 3.9; MR+S Chapters 2.1-2.2

3.
Minimum Edit Distance: J+M Chapter 3.11

4.
Language Modeling: J+M Chapter 4

5.
Spelling Correction: J+M Chapters 5.9, Peter Norvig (2007) How to Write a Spelling Corrector




--------------------------------------------------------------------------------


Levenshtein distance (Minimum Edit Distance)


In information theory and computer science, the Levenshtein distance is a string metric for measuring the amount of difference between two sequences. The term edit distance is often used to refer specifically to Levenshtein distance.


The Levenshtein distance between two strings is defined as the minimum number of edits needed to transform one string into the other, with the allowable edit operations being insertion, deletion, or substitution of a single character. It is named after Vladimir Levenshtein, who considered this distance in 1965.


For example, the Levenshtein distance between "kitten" and "sitting" is 3, since the following three edits change one into the other, and there is no way to do it with fewer than three edits:
1.
kitten → sitten (substitution of 's' for 'k')

2.
sitten → sittin (substitution of 'i' for 'e')

3.
sittin → sitting (insertion of 'g' at the end).




--------------------------------------------------------------------------------



Principle Of Maximum Entropy


In Bayesian probability theory, the principle of maximum entropy is an axiom. It states that, subject to precisely stated prior data, which must be a proposition that expresses testable information, the probability distribution which best represents the current state of knowledge is the one with largest information theoretical entropy.


Let some precisely stated prior data or testable information about a probability distribution function be given. Consider the set of all trial probability distributions that encode the prior data. Of those, the one that maximizes the information entropy is the proper probability distribution under the given prior data.




--------------------------------------------------------------------------------


Sentiment analysis


Sentiment analysis or opinion mining refers to the application of natural language processing, computational linguistics, and text analytics to identify and extract subjective information in source materials.

Generally speaking, sentiment analysis aims to determine the attitude of a speaker or a writer with respect to some topic or the overall contextual polarity of a document. 

The attitude may be his or her judgement or evaluation (see appraisal theory), affective state (that is to say, the emotional state of the author when writing), or the intended emotional communication (that is to say, the emotional effect the author wishes to have on the reader).




--------------------------------------------------------------------------------


