1.5 Software requirements
This course assumes you have access to the Natural Language Toolkit (NLTK) either
on your own computer or at your institution. The NLTK can be freely downloaded
and it is strongly recommended that you install it on your own machine: Windows,
Mac OSX and Linux distributions are available from http://nltk.org (last visited
April 10th 2013) and some distributions of Linux have it in their package/software
managers. Full instructions are available at the cited website along with details of
associated packages which should also be installed, including Python itself which is
also freely available. Once you have installed the software you should also download
the required datasets as explained in the textbook (Bird et al., 2009, p. 3).
You should check the NLTK website to determine what versions of Python are
supported. Current stable releases of NLTK are compatible with Python 2.6 and 2.7.
A version supporting Python 3 is under development and may be available for
testing by the time you read this guide (as of April 2013).


%=================================================================================%
1.6.3 Chapter 4: Computational tools for text analysis
The previous chapter introduced some relatively superficial techniques for language
analysis such as concordancing and collocations. This chapter covers some
fundamental operations in text analysis:
tokenisation: breaking up a character string into words, punctuation marks and
other meaningful expressions;
stemming: removing affixes from words, e.g. mean+ing, distribut+ion;
tagging: associating each word in a text with a grammatical category or part of
speech.

%==========================================================%
1.6.4 Chapter 5: Statistically-based techniques for text analysis
Statistical and probabilistic methods are pervasive in modern computational
linguistics. These methods generally do not aim at complete understanding or
analysis of a text, but at producing reliable answers to well-defined problems such as
sentiment analysis, topic detection or recognising named entities and relations
between them in a text.

%==========================================================%
1.6.5 Chapter 6: Analysing sentences: syntax and parsing
This chapter resumes the discussion of natural language syntax that was introduced
in Chapter 2, concentrating on context-free grammar formalisms and various ways
they need to be modified and extended beyond the model that was presented in that
chapter. Formal grammars do not encode any kind of processing strategy but simply
provide a declarative specification of the well-formed sentences in a language.
Parsers are computer programs that use grammar rules to analyse sentences, and
this chapter introduces some fundamental approaches to syntactic parsing.

%==========================================================%
1.6.6 Appendices
The Appendices include:
A. A bibliography listing all works referenced in the subject guide, including
publication details and ISBNs.
B. A glossary of technical terms used in this subject guide.
C. Answers to selected activities.
D. A trace of a recursive descent parse as described in Chapter 6 of the subject guide.
E. A sample examination paper with guidelines on how to answer questions.

%==========================================================%
