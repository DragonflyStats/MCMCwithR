Lexical analysis
From Wikipedia, the free encyclopedia
In computer science, lexical analysis is the process of converting a sequence of characters into a sequence of tokens, i.e. meaningful character strings. A program or function that performs lexical analysis is called a lexical analyzer, lexer, tokenizer,[1] or scanner, though "scanner" is also used for the first stage of a lexer. A lexer is generally combined with a parser, which together analyze the syntax of programming languages, such as in compilers, but also HTML parsers in web browsers, among other examples.
Strictly speaking, a lexer is itself a kind of parser – the syntax of some programming languages is divided into two pieces: the lexical syntax (token structure), which is processed by the lexer; and the phrase syntax, which is processed by the parser. The lexical syntax is usually a regular language, whose alphabet consists of the individual characters of the source code text. The phrase syntax is usually a context-free language, whose alphabet consists of the tokens produced by the lexer. While this is a common separation, alternatively, a lexer can be combined with the parser in scannerless parsing.
